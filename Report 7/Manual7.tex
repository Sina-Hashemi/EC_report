%%%%%%%%%%%%%%%%%%%%%%%%%%%%%%%%%%%%%%%%%
% Cleese Assignment (For Students)
% LaTeX Template
% Version 2.0 (27/5/2018)
%
% This template originates from:
% http://www.LaTeXTemplates.com
%
% Author:
% Vel (vel@LaTeXTemplates.com)
%
% License:
% CC BY-NC-SA 3.0 (http://creativecommons.org/licenses/by-nc-sa/3.0/)
% 
%%%%%%%%%%%%%%%%%%%%%%%%%%%%%%%%%%%%%%%%%

%----------------------------------------------------------------------------------------
%	PACKAGES AND OTHER DOCUMENT CONFIGURATIONS
%----------------------------------------------------------------------------------------

\documentclass[11pt]{article}
\usepackage{float}

%\usepackage[printwatermark]{xwatermark}
%\newwatermark[allpages,color=gray!50,angle=45,scale=2.5,xpos=-5,ypos=-5]{Mohammad Hadi}

\input{structure.tex} % Include the file specifying the document structure and custom commands

%----------------------------------------------------------------------------------------
%	ASSIGNMENT INFORMATION
%----------------------------------------------------------------------------------------

% Required
\newcommand{\assignmentQuestionName}{Experiment} % The word to be used as a prefix to question numbers; example alternatives: Problem, Exercise
\newcommand{\assignmentClass}{Electrical Circuits Lab (Taught by Mohammad Hadi)\\Manual 7 (Due on DDD.,\ mmm.\ dd,\ yyyy)} % Course (Lecturer)\\Assignment (Due date)
\newcommand{\assignmentTitle}{} % Assignment title or name
\newcommand{\assignmentAuthorName}{Sina Hashemi \& M.Mahdi Shokrzade\\402102668 - 402101985} % Student name\\Student number
%----------------------------------------------------------------------------------------

\begin{document}
\textbf{Frequency response is a measurable concept that can be used to experimentally and analytically describe a circuit. In this experiment, you become familiar with the frequency response of first- and second-order circuits.
}
%----------------------------------------------------------------------------------------
%	TITLE PAGE
%----------------------------------------------------------------------------------------

\assignmentSection{Mandatory Experiments}

%----------------------------------------------------------------------------------------
%	QUESTION 1
%----------------------------------------------------------------------------------------

\begin{question}

    \questiontext{Build the circuit shown in Fig. \ref{fig:cir1} on a breadboard. Instead of using a physical switch, connect or disconnect the supply wire.}

    \begin{figure}[H]
        \centering
        \includegraphics[scale=1.2,angle=0]{Fig/cir1.pdf}
        \caption{An RC circuit.} \label{fig:cir1}
    \end{figure}

    %--------------------------------------------
    \begin{subquestion}{Connect and disconnect the supplu wire to see the capacitor voltage response on the oscilloscope screen. Discuss the results.}
        \answer{}
    \end{subquestion}

    %--------------------------------------------
    \begin{subquestion}{Measure the time constant of the circuit and compare it to its analytical counterpart.}
        \answer{}
    \end{subquestion}

    %--------------------------------------------
    \begin{subquestion}{Decrease the supply voltage to $5$ V and redo the previous parts.}
        \answer{}
    \end{subquestion}


\end{question}

%----------------------------------------------------------------------------------------
%	QUESTION 2
%----------------------------------------------------------------------------------------

\begin{question}

    \questiontext{Build the circuit shown in Fig. \ref{fig:cir2} on a breadboard.}

    \begin{figure}[H]
        \centering
        \includegraphics[scale=1.2,angle=0]{Fig/cir2.pdf}
        \caption{A lowpass RC circuit.} \label{fig:cir2}
    \end{figure}

    %--------------------------------------------
    \begin{subquestion}{Apply a $1$-V $300$-Hz square wave to the input. See the input and output voltages simultaneously on the oscilloscope screen. Interpret the observations. Obtain the time constant of the circuit.}
        \answer{}
    \end{subquestion}

    %--------------------------------------------
    \begin{subquestion}{Apply a $1$-V sine wave to the input and change its frequency to measure the frequency response of the circuit using interpolation. See the input and output voltages simultaneously on the oscilloscope and discuss the filtering behavior of the circuit.}
        \answer{}
    \end{subquestion}


\end{question}

%----------------------------------------------------------------------------------------
%	QUESTION 3
%----------------------------------------------------------------------------------------

\begin{question}

    \questiontext{Build the circuit shown in Fig. \ref{fig:cir3} on a breadboard.}

    \begin{figure}[H]
        \centering
        \includegraphics[scale=1.2,angle=0]{Fig/cir3.pdf}
        \caption{A highpass RC circuit.} \label{fig:cir3}
    \end{figure}

    %--------------------------------------------
    \begin{subquestion}{Apply a $1$-V $300$-Hz square wave to the input. See the input and output voltages simultaneously on the oscilloscope screen. Interpret the observations. Obtain the time constant of the circuit.}
        \answer{}
    \end{subquestion}

    %--------------------------------------------
    \begin{subquestion}{Apply a $1$-V sine wave to the input and change its frequency to measure the frequency response of the circuit using interpolation. See the input and output voltages simultaneously on the oscilloscope and discuss the filtering behavior of the circuit.}
        \answer{}
    \end{subquestion}


\end{question}

%----------------------------------------------------------------------------------------
%	QUESTION 4
%----------------------------------------------------------------------------------------

\begin{question}

    \questiontext{Build the circuit shown in Fig. \ref{fig:cir4} on a breadboard.}

    \begin{figure}[H]
        \centering
        \includegraphics[scale=1.2,angle=0]{Fig/cir4.pdf}
        \caption{A bandpass RLC circuit.} \label{fig:cir4}
    \end{figure}

    %--------------------------------------------
    \begin{subquestion}{Apply a $1$-V sine wave to the input and change its frequency from $100$ Hz to $10$ kHz to measure the frequency response of the circuit using interpolation. See the input and output voltages simultaneously on the oscilloscope and discuss the filtering behavior of the circuit.}
        \answer{}
    \end{subquestion}

    %--------------------------------------------
    \begin{subquestion}{Apply a $1$-V square wave to the input and sweep its frequency from $100$ Hz to $10$ kHz. See the input and output voltages simultaneously on an oscilloscope screen. Interpret the observations.}
        \answer{}
    \end{subquestion}

\end{question}


\assignmentSection{Bonus Experiments}

%----------------------------------------------------------------------------------------
%	QUESTION 5
%----------------------------------------------------------------------------------------

\begin{question}

    \questiontext{The circuit shown in Fig. \ref{fig:Q5} is called Sallen active lowpass filter, where the triangle abstracts an op-amp amplification circuit with the gain $K$.}

    \begin{figure}[H]
        \begin{center}
            \includegraphics[scale=0.3]{Fig/Q5.png}
            \caption{\label{fig:Q5} Sallen active lowpass filter.}
        \end{center}
    \end{figure}

    %--------------------------------------------
    \begin{subquestion}{Calculate the frequency response of the active filter circuit.}
        \answer{}
    \end{subquestion}
    %--------------------------------------------
    \begin{subquestion}{How can the amplifier part be implemented using op-amps?}
        \answer{}
    \end{subquestion}
    %--------------------------------------------
    \begin{subquestion}{What are the advantages of the amplification part? Does it have any impact on the filtering response?}
        \answer{}
    \end{subquestion}
    %--------------------------------------------
    \begin{subquestion}{What are the advantages and disadvantages of such an active filter?}
        \answer{}
    \end{subquestion}

\end{question}


%----------------------------------------------------------------------------------------
%	QUESTION 6
%----------------------------------------------------------------------------------------
\begin{question}
    \questiontext{The circuit shown in Fig. \ref{fig:Q6} is called biquad active filter. The triangles denote two amplifiers with the gains $-1$ and $2$. The amplifiers may be implemented using inverting and non-inverting op-amp circuits. The admittances $Y_1$, $Y_2$, $Y_3$ and $Y_4$ can be replaced by series or parallel RC circuits. A sample customized configuration is shown in Fig. \ref{fig:Q6_1}. Depending on the configuration, the circuit provides various filtering responses.
    }
    \begin{figure}[H]
        \begin{center}
            \includegraphics[scale=0.5]{Fig/Q6.png}
            \caption{\label{fig:Q6} Biquad active filter.}
        \end{center}
    \end{figure}
    \begin{figure}[H]
        \begin{center}
            \includegraphics[scale=0.5]{Fig/Q6_1.png}
            \caption{\label{fig:Q6_1} A sample customized realization of the biquad active filter.}
        \end{center}
    \end{figure}


    %--------------------------------------------
    \begin{subquestion}{Simulate the circuit in PSpice and investigate the filtering response of the circuit for various configurations of $Y_1$, $Y_2$, $Y_3$ and $Y_4$. Especially, demonstrate how the biquad filter can have lowpass, highpass, bandpass, and bandstop frequency responses.}
        \answer{}
    \end{subquestion}
    %--------------------------------------------
    \begin{subquestion}{What is an all-pass filter and how can it be implemented using a biquad?}
        \answer{}
    \end{subquestion}

\end{question}


%----------------------------------------------------------------------------------------
%	QUESTION 7
%----------------------------------------------------------------------------------------

\begin{question}

    \questiontext{Return your work report by filling the \LaTeX template of the manual. Include useful and high-quality images to make the report more readable and understandable.}

\end{question}

%----------------------------------------------------------------------------------------

\end{document}
