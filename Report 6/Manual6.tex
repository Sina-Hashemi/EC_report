%%%%%%%%%%%%%%%%%%%%%%%%%%%%%%%%%%%%%%%%%
% Cleese Assignment (For Students)
% LaTeX Template
% Version 2.0 (27/5/2018)
%
% This template originates from:
% http://www.LaTeXTemplates.com
%
% Author:
% Vel (vel@LaTeXTemplates.com)
%
% License:
% CC BY-NC-SA 3.0 (http://creativecommons.org/licenses/by-nc-sa/3.0/)
% 
%%%%%%%%%%%%%%%%%%%%%%%%%%%%%%%%%%%%%%%%%

%----------------------------------------------------------------------------------------
%	PACKAGES AND OTHER DOCUMENT CONFIGURATIONS
%----------------------------------------------------------------------------------------

\documentclass[11pt]{article}
\usepackage{float}

%\usepackage[printwatermark]{xwatermark}
%\newwatermark[allpages,color=gray!50,angle=45,scale=2.5,xpos=-5,ypos=-5]{Mohammad Hadi}

\input{structure.tex} % Include the file specifying the document structure and custom commands

%----------------------------------------------------------------------------------------
%	ASSIGNMENT INFORMATION
%----------------------------------------------------------------------------------------

% Required
\newcommand{\assignmentQuestionName}{Experiment} % The word to be used as a prefix to question numbers; example alternatives: Problem, Exercise
\newcommand{\assignmentClass}{Electrical Circuits Lab (Taught by Mohammad Hadi)\\Manual 6 (Due on DDD.,\ mmm.\ dd,\ yyyy)} % Course (Lecturer)\\Assignment (Due date)
\newcommand{\assignmentTitle}{} % Assignment title or name
\newcommand{\assignmentAuthorName}{Sina Hashemi \& M.Mahdi Shokrzade\\402102668 - 402101985} % Student name\\Student number
%----------------------------------------------------------------------------------------

\begin{document}
\textbf{Op-amps are versatile elements used to implement various circuits such as amplifiers, comparators, filters, and son on. In this experiment, you become familiar with a typical op-amp and its common applications.
}
%----------------------------------------------------------------------------------------
%	TITLE PAGE
%----------------------------------------------------------------------------------------

\assignmentSection{Mandatory Experiments}

%----------------------------------------------------------------------------------------
%	QUESTION 1
%----------------------------------------------------------------------------------------

\begin{question}

    \questiontext{Build the circuit shown in Fig. \ref{fig:cir1} using an op-amp comparator module. Create a pair of $\pm 18$ V voltages and connect them to the supply connectors of the module.}

    \begin{figure}[H]
        \centering
        \includegraphics[scale=1.2,angle=0]{Fig/cir1.pdf}
        \caption{An op-amp as a comparator.} \label{fig:cir1}
    \end{figure}

    %--------------------------------------------
    \begin{subquestion}{Set $V_{s1}=0$ or equivalently, connect the inverting input of the op-amp to the ground. Apply a $1$-V $1$-kHz sine voltage $v_{s2}(t)$ to the non-inverting input. Watch the the output voltage and the non-inverting input voltage of the op-amp simultaneously on the oscilloscope. Interpret the results. Change the sine wave to a triangle wave and observe the results.}
        \answer{}
    \end{subquestion}

    %--------------------------------------------
    \begin{subquestion}{Set $V_{s1}=\pm 0.5$ V and repeat the previous part.}
        \answer{}
    \end{subquestion}

    %--------------------------------------------
    \begin{subquestion}{Swap the input voltages to the op-amp and redo the previous parts.}
        \answer{}
    \end{subquestion}


\end{question}

%----------------------------------------------------------------------------------------
%	QUESTION 2
%----------------------------------------------------------------------------------------

\begin{question}

    \questiontext{Build the circuit shown in Fig. \ref{fig:cir2} using an op-amp comparator module. Create a pair of $\pm 18$ V voltages and connect them to the supply connectors of the module as well as to the fixed legs of the potentiometer.}

    \begin{figure}[H]
        \centering
        \includegraphics[scale=1.2,angle=0]{Fig/cir2.pdf}
        \caption{An op-amp as a comparator along with a potentiometer as a voltage divider.} \label{fig:cir2}
    \end{figure}

    %--------------------------------------------
    \begin{subquestion}{Apply a $1$-V $1$-kHz sine voltage $v_{s2}(t)$ to the non-inverting input. Watch the the output voltage and the non-inverting input voltage of the op-amp simultaneously on the oscilloscope. Turn the knob of the potentiometer and observe the results.}
        \answer{}
    \end{subquestion}

    %--------------------------------------------
    \begin{subquestion}{Repeat the previous part for a triangle wave.}
        \answer{}
    \end{subquestion}


\end{question}

%----------------------------------------------------------------------------------------
%	QUESTION 3
%----------------------------------------------------------------------------------------

\begin{question}

    \questiontext{Build the circuit shown in Fig. \ref{fig:cir3} using an op-amp inverting amplifier module. Create a pair of $\pm 18$ V voltages and connect them to the supply connectors of the module.}

    \begin{figure}[H]
        \centering
        \includegraphics[scale=1.2,angle=0]{Fig/cir3.pdf}
        \caption{Inverting amplifier.} \label{fig:cir3}
    \end{figure}

    %--------------------------------------------
    \begin{subquestion}{Apply a $0.5$-V $1$-kHz sine voltage $v_{s}(t)$ to the input of the amplifier. Watch the the output and input voltages of the amplifier simultaneously on the oscilloscope. Calculate the gain of the amplifier.}
        \answer{}
    \end{subquestion}

    %--------------------------------------------
    \begin{subquestion}{Devise an experiment to measure the input and output resistance of the amplifier module.}
        \answer{}
    \end{subquestion}

    %--------------------------------------------
    \begin{subquestion}{Increase the amplitude of the input $1$-kHz sine wave and record your observations. Interpret and discuss the results.}
        \answer{}
    \end{subquestion}

    %--------------------------------------------
    \begin{subquestion}{Increase the frequency of the input $0.5$-V sine wave and record your observations. Interpret and discuss the results.}
        \answer{}
    \end{subquestion}

\end{question}


%----------------------------------------------------------------------------------------
%	QUESTION 4
%----------------------------------------------------------------------------------------

\begin{question}

    \questiontext{Build the circuit shown in Fig. \ref{fig:cir4} using an op-amp non-inverting amplifier module. Create a pair of $\pm 18$ V voltages and connect them to the supply connectors of the module.}

    \begin{figure}[H]
        \centering
        \includegraphics[scale=1.2,angle=0]{Fig/cir4.pdf}
        \caption{Non-inverting amplifier.} \label{fig:cir4}
    \end{figure}

    %--------------------------------------------
    \begin{subquestion}{Apply a $0.5$-V $1$-kHz sine voltage $v_{s}(t)$ to the input of the amplifier. Watch the the output and input voltages of the amplifier simultaneously on the oscilloscope. Calculate the gain of the amplifier.}
        \answer{}
    \end{subquestion}

    %--------------------------------------------
    \begin{subquestion}{Measure the input and output resistance of the amplifier module experimentally.}
        \answer{}
    \end{subquestion}

    %--------------------------------------------
    \begin{subquestion}{Increase the amplitude of the input $1$-kHz sine wave and record your observations. Interpret and discuss the results.}
        \answer{}
    \end{subquestion}

    %--------------------------------------------
    \begin{subquestion}{Increase the frequency of the input $0.5$-V sine wave and record your observations. Interpret and discuss the results.}
        \answer{}
    \end{subquestion}

\end{question}

%----------------------------------------------------------------------------------------
%	QUESTION 5
%----------------------------------------------------------------------------------------

\begin{question}

    \questiontext{Cascade an inverting amplifier and a non-inverting amplifier as shown in Fig. \ref{fig:cir5}}.

    \begin{figure}[H]
        \centering
        \includegraphics[scale=1.2,angle=0]{Fig/cir5.pdf}
        \caption{Cascade of two amplifiers.} \label{fig:cir5}
    \end{figure}

    %--------------------------------------------
    \begin{subquestion}{Apply a $100$-mV $1$-kHz sine voltage $v_{s}(t)$ to the input of the cascaded amplifiers. Watch the the output and input voltages of the cascaded amplifiers simultaneously on the oscilloscope. Calculate the overall gain of the cascaded amplifiers.}
        \answer{}
    \end{subquestion}

    %--------------------------------------------
    \begin{subquestion}{Swap the order of the amplifiers and repeat the previous part. Is there any difference between the measured gains in the two experiments? Explain.}
        \answer{}
    \end{subquestion}

\end{question}


\assignmentSection{Bonus Experiments}

%----------------------------------------------------------------------------------------
%	QUESTION 6
%----------------------------------------------------------------------------------------

\begin{question}

    \questiontext{In a circuit design, we need to cascade an inverting and a non-inverting amplifier to get the overall gain of $G_{tot}=G_{inv}G_{nnv}$. }

    %-------------------------------------------------------------
    \begin{subquestion}{From analytical point of view, is there any difference to change the order of the cascaded amplifiers?}
        \answer{}
    \end{subquestion}

    %-------------------------------------------------------------
    \begin{subquestion}{From practical point of view, is there any difference to change the order of the cascaded amplifiers? Justify your answer using PSpice simulation.}
        \answer{}
    \end{subquestion}

\end{question}


%----------------------------------------------------------------------------------------
%	QUESTION 7
%----------------------------------------------------------------------------------------
\begin{question}

    \questiontext{Op-amps usually need a pair of positive and negative DC supply voltages $\pm V_s$.}

    %-------------------------------------------------------------
    \begin{subquestion}{What happens if the absolute values of the supply voltages differ? }
        \answer{}
    \end{subquestion}

    %-------------------------------------------------------------
    \begin{subquestion}{Is it possible to use an op-amp with the supply voltages $0$ and $+V_s$? Explain.}
        \answer{}
    \end{subquestion}

\end{question}


%----------------------------------------------------------------------------------------
%	QUESTION 8
%----------------------------------------------------------------------------------------

\begin{question}

    \questiontext{Return your work report by filling the \LaTeX template of the manual. Include useful and high-quality images to make the report more readable and understandable.}

\end{question}

%----------------------------------------------------------------------------------------

\end{document}
