%%%%%%%%%%%%%%%%%%%%%%%%%%%%%%%%%%%%%%%%%
% Cleese Assignment (For Students)
% LaTeX Template
% Version 2.0 (27/5/2018)
%
% This template originates from:
% http://www.LaTeXTemplates.com
%
% Author:
% Vel (vel@LaTeXTemplates.com)
%
% License:
% CC BY-NC-SA 3.0 (http://creativecommons.org/licenses/by-nc-sa/3.0/)
% 
%%%%%%%%%%%%%%%%%%%%%%%%%%%%%%%%%%%%%%%%%

%----------------------------------------------------------------------------------------
%	PACKAGES AND OTHER DOCUMENT CONFIGURATIONS
%----------------------------------------------------------------------------------------

\documentclass[11pt]{article}
\usepackage{float}

%\usepackage[printwatermark]{xwatermark}
%\newwatermark[allpages,color=gray!50,angle=45,scale=2.5,xpos=-5,ypos=-5]{Mohammad Hadi}

\input{structure.tex} % Include the file specifying the document structure and custom commands

%----------------------------------------------------------------------------------------
%	ASSIGNMENT INFORMATION
%----------------------------------------------------------------------------------------

% Required
\newcommand{\assignmentQuestionName}{Experiment} % The word to be used as a prefix to question numbers; example alternatives: Problem, Exercise
\newcommand{\assignmentClass}{Electrical Circuits Lab (Taught by Mohammad Hadi)\\Manual 6 (Due on DDD.,\ mmm.\ dd,\ yyyy)} % Course (Lecturer)\\Assignment (Due date)
\newcommand{\assignmentTitle}{} % Assignment title or name
\newcommand{\assignmentAuthorName}{Sina Hashemi \& M.Mahdi Shokrzade\\402102668 - 402101985} % Student name\\Student number
%----------------------------------------------------------------------------------------
\newcommand{\PicScale}{0.2}

\begin{document}
\textbf{Op-amps are versatile elements used to implement various circuits such as amplifiers, comparators, filters, and son on. In this experiment, you become familiar with a typical op-amp and its common applications.
}
%----------------------------------------------------------------------------------------
%	TITLE PAGE
%----------------------------------------------------------------------------------------

\assignmentSection{Mandatory Experiments}

%----------------------------------------------------------------------------------------
%	QUESTION 1
%----------------------------------------------------------------------------------------

\begin{question}

    \questiontext{Build the circuit shown in Fig. \ref{fig:cir1} using an op-amp comparator module. Create a pair of $\pm 18$ V voltages and connect them to the supply connectors of the module.}

    \begin{figure}[H]
        \centering
        \includegraphics[scale=1.2,angle=0]{Fig/cir1.pdf}
        \caption{An op-amp as a comparator.} \label{fig:cir1}
    \end{figure}

    %--------------------------------------------
    \begin{subquestion}{Set $V_{s1}=0$ or equivalently, connect the inverting input of the op-amp to the ground. Apply a $1$-V $1$-kHz sine voltage $v_{s2}(t)$ to the non-inverting input. Watch the the output voltage and the non-inverting input voltage of the op-amp simultaneously on the oscilloscope. Interpret the results. Change the sine wave to a triangle wave and observe the results.}
        \answer{
            \begin{figure}[H]
                \centering
                \includegraphics[scale=0.1,angle=0]{Fig/1.jpeg}
                \includegraphics[scale=0.1,angle=0]{Fig/2.jpeg}
                \caption{Feeding Op-Amp using $+18V$ and $-18V$.}
            \end{figure}
            \begin{figure}[H]
                \centering
                \includegraphics[scale=\PicScale,angle=0]{Fig/3.jpeg}
                \caption{The circuit.}
            \end{figure}
            \begin{figure}[H]
                \centering
                \includegraphics[scale=\PicScale,angle=0]{Fig/4.jpeg}
                \caption{oscilloscope's screen for sine wace.}
            \end{figure}
            We know $V_d = V_+ - V_- = V_{\sin} - 0$.
            The op-amp boosts the input voltage, but since the output voltage exceeds +E at some points, the output voltage does not exceed this number. For the same reason, the output voltage does not decrease from -E.
        }
    \end{subquestion}

    %--------------------------------------------
    \begin{subquestion}{Set $V_{s1}=\pm 0.5$ V and repeat the previous part.}
        \answer{
            \begin{figure}[H]
                \centering
                \includegraphics[scale=0.1,angle=0]{Fig/5.jpeg}
                \includegraphics[scale=0.1,angle=0]{Fig/6.jpeg}
                \includegraphics[scale=0.1,angle=0]{Fig/8.jpeg}
                \caption{The circuit used for making $\pm 0.5V$ using potentiometer.}
            \end{figure}
            We know $V_d = V_+ - V_- = V_{\sin} - 0.5$. So for +E voltage we have $V_d > 0 \Rightarrow V_{\sin} > 0.5$ and $V_d < 0 \Rightarrow V_{\sin} < 0.5$.
            \begin{figure}[H]
                \centering
                \includegraphics[scale=\PicScale,angle=0]{Fig/7.jpeg}
                \caption{oscilloscope's screen for $+0.5V$.}
            \end{figure}
            We know $V_d = V_+ - V_- = V_{\sin} + 0.5$. So for +E voltage we have $V_d > 0 \Rightarrow V_{\sin} > -0.5$ and $V_d < 0 \Rightarrow V_{\sin} < -0.5$.
            \begin{figure}[H]
                \centering
                \includegraphics[scale=\PicScale,angle=0]{Fig/9.jpeg}
                \caption{oscilloscope's screen for $-0.5V$.}
            \end{figure}
            The above notes also applies to the triangular wave.
            \begin{figure}[H]
                \centering
                \includegraphics[scale=\PicScale,angle=0]{Fig/51.jpeg}
                \caption{oscilloscope's screen for $+0.5V$ for triangle wave.}
            \end{figure}
            \begin{figure}[H]
                \centering
                \includegraphics[scale=\PicScale,angle=0]{Fig/52.jpeg}
                \caption{oscilloscope's screen for $-0.5V$ for triangle wave.}
            \end{figure}
            }
        \end{subquestion}

        %--------------------------------------------
        \begin{subquestion}{Swap the input voltages to the op-amp and redo the previous parts.}
            \answer{
                \begin{figure}[H]
                    \centering
                    \includegraphics[scale=\PicScale,angle=0]{Fig/10.jpeg}
                    \caption{The circuit.}
                \end{figure}
                We know $V_d = V_+ - V_- = 0 - V_{\sin}$.
                \begin{figure}[H]
                    \centering
                    \includegraphics[scale=\PicScale,angle=0]{Fig/11.jpeg}
                    \caption{oscilloscope's screen for $0V$.}
                \end{figure}
                We know $V_d = V_+ - V_- = 0.5 - V_{\sin}$. So for +E voltage we have $V_d > 0 \Rightarrow V_{\sin} < 0.5$ and $V_d < 0 \Rightarrow V_{\sin} > 0.5$.
                \begin{figure}[H]
                    \centering
                    \includegraphics[scale=\PicScale,angle=0]{Fig/12.jpeg}
                    \caption{oscilloscope's screen for $+0.5V$.}
                We know $V_d = V_+ - V_- = -0.5 - V_{\sin}$. So for +E voltage we have $V_d > 0 \Rightarrow V_{\sin} < -0.5$ and $V_d < 0 \Rightarrow V_{\sin} > -0.5$.
                \end{figure}
                \begin{figure}[H]
                    \centering
                    \includegraphics[scale=\PicScale,angle=0]{Fig/13.jpeg}
                    \caption{oscilloscope's screen for $-0.5V$.}
                \end{figure}
                The above notes also applies to the triangular wave.
                \begin{figure}[H]
                    \centering
                    \includegraphics[scale=\PicScale,angle=0]{Fig/53.jpeg}
                    \caption{oscilloscope's screen for $0V$ for triangle wave.}
                \end{figure}
                \begin{figure}[H]
                    \centering
                    \includegraphics[scale=\PicScale,angle=0]{Fig/54.jpeg}
                    \caption{oscilloscope's screen for $+0.5V$ for triangle wave.}
                \end{figure}
        }
    \end{subquestion}


\end{question}

%----------------------------------------------------------------------------------------
%	QUESTION 2
%----------------------------------------------------------------------------------------

\begin{question}

    \questiontext{Build the circuit shown in Fig. \ref{fig:cir2} using an op-amp comparator module. Create a pair of $\pm 18$ V voltages and connect them to the supply connectors of the module as well as to the fixed legs of the potentiometer.}

    \begin{figure}[H]
        \centering
        \includegraphics[scale=1.2,angle=0]{Fig/cir2.pdf}
        \caption{An op-amp as a comparator along with a potentiometer as a voltage divider.} \label{fig:cir2}
    \end{figure}

    %--------------------------------------------
    \begin{subquestion}{Apply a $1$-V $1$-kHz sine voltage $v_{s2}(t)$ to the non-inverting input. Watch the the output voltage and the non-inverting input voltage of the op-amp simultaneously on the oscilloscope. Turn the knob of the potentiometer and observe the results.}
        \answer{
            \begin{figure}[H]
                \centering
                \includegraphics[scale=\PicScale,angle=0]{Fig/14.jpeg}
                \caption{The circuit.}
            \end{figure}
            \begin{figure}[H]
                \centering
                \includegraphics[scale=0.08,angle=0]{Fig/15.png}
                \includegraphics[scale=0.08,angle=0]{Fig/16.png}
                \includegraphics[scale=0.08,angle=0]{Fig/17.png}
                \caption{oscilloscope's screen.}
            \end{figure}
            As you can see, there is a voltage division between the two output sides of the potentiometer and a variable voltage (adjustable by us) enters the inverting base of the op-amp. We knows $V_d = V_+ - V_- = V_{\sin} - V_{potentiometer}$. As a result, by changing the voltage of the potentiometer and, the output will be the different for each voltage of the potentiometer, which you can see in the above images in three states.
        }
    \end{subquestion}

    %--------------------------------------------
    \begin{subquestion}{Repeat the previous part for a triangle wave.}
        \answer{
            \begin{figure}[H]
                \centering
                \includegraphics[scale=0.08,angle=0]{Fig/18.png}
                \includegraphics[scale=0.08,angle=0]{Fig/19.png}
                \includegraphics[scale=0.08,angle=0]{Fig/20.png}
                \caption{oscilloscope's screen.}
            \end{figure}
        }
    \end{subquestion}


\end{question}

%----------------------------------------------------------------------------------------
%	QUESTION 3
%----------------------------------------------------------------------------------------

\begin{question}

    \questiontext{Build the circuit shown in Fig. \ref{fig:cir3} using an op-amp inverting amplifier module. Create a pair of $\pm 18$ V voltages and connect them to the supply connectors of the module.}

    \begin{figure}[H]
        \centering
        \includegraphics[scale=1.2,angle=0]{Fig/cir3.pdf}
        \caption{Inverting amplifier.} \label{fig:cir3}
    \end{figure}

    %--------------------------------------------
    \begin{subquestion}{Apply a $0.5$-V $1$-kHz sine voltage $v_{s}(t)$ to the input of the amplifier. Watch the the output and input voltages of the amplifier simultaneously on the oscilloscope. Calculate the gain of the amplifier.}
        \answer{
            \begin{figure}[H]
                \centering
                \includegraphics[scale=\PicScale,angle=0]{Fig/21.jpeg}
                \caption{The circuit.}
            \end{figure}
            \begin{figure}[H]
                \centering
                \includegraphics[scale=\PicScale,angle=0]{Fig/22.jpeg}
                \caption{oscilloscope's screen for $0.5V$ voltage.}
            \end{figure}
            \begin{figure}[H]
                \centering
                \includegraphics[scale=\PicScale,angle=0]{Fig/23.jpeg}
                \caption{oscilloscope's screen for $5mV$ voltage which used for Calculating Op-Amp's gain.}
            \end{figure}
            We use the data from the second picture because the op amp is not saturated.
            \[
                Gain = \frac{V_{out}}{V_{in}} = \frac{2 \times 100mV}{1.5 \times 5mV} = 26.6
            \]
        }
    \end{subquestion}

    %--------------------------------------------
    \begin{subquestion}{Devise an experiment to measure the input and output resistance of the amplifier module.}
        \answer{
            For finding $R_{in}$ we connect a $R_s$ after $V_s$ and before $V_1$, we have:
            \[
                V_1 = \frac{R_{in}}{R_{in} + R_s} V_s
            \]
            \begin{figure}[H]
                \centering
                \includegraphics[scale=\PicScale,angle=0]{Fig/38.jpeg}
                \caption{Circuit for $R_{in}$.}
            \end{figure}
            \begin{figure}[H]
                \centering
                \includegraphics[scale=\PicScale,angle=0]{Fig/39.jpeg}
                \caption{oscilloscope's screen for $R_{in}$.}
            \end{figure}
            \[
                \frac{R_{in}}{R_{in} + 10k} = \frac{13}{17} \Rightarrow R_{in} = 32.5k\Omega
            \]
            For finding $R_{out}$ We connect a resistor $R_L$. At first $R_L \to \infty$ so we find $AV_d$ and then we set $R_L = xk\Omega$, thus:
            \[
                V_2 = AV_d \frac{R_L}{R_L + R_{out}}
            \]
            \begin{figure}[H]
                \centering
                \includegraphics[scale=\PicScale,angle=0]{Fig/40.jpeg}
                \caption{Circuit for $R_{out}$ when $R_L \to \infty$.}
            \end{figure}
            \begin{figure}[H]
                \centering
                \includegraphics[scale=\PicScale,angle=0]{Fig/41.jpeg}
                \caption{oscilloscope's screen for $R_{in}$ when $R_L \to \infty$.}
            \end{figure}
            \begin{figure}[H]
                \centering
                \includegraphics[scale=\PicScale,angle=0]{Fig/42.jpeg}
                \caption{Circuit for $R_{out}$ when $R_L = 1k\Omega$.}
            \end{figure}
            \begin{figure}[H]
                \centering
                \includegraphics[scale=\PicScale,angle=0]{Fig/43.jpeg}
                \caption{oscilloscope's screen for $R_{in}$ when $R_L = 1k\Omega$.}
            \end{figure}
            \[
                \frac{1k}{1k + R_{out}} = \frac{7}{17} \Rightarrow R_{out} = 1.4k\Omega
            \]
        }
    \end{subquestion}

    %--------------------------------------------
    \begin{subquestion}{Increase the amplitude of the input $1$-kHz sine wave and record your observations. Interpret and discuss the results.}
        \answer{
            \begin{figure}[H]
                \centering
                \includegraphics[scale=0.08,angle=0]{Fig/24.png}
                \includegraphics[scale=0.08,angle=0]{Fig/25.png}
                \includegraphics[scale=0.08,angle=0]{Fig/26.png}
                \caption{oscilloscope's screen for different input amplitude.}
            \end{figure}
            As you can see, the op-amp starts to saturate as the amplitude increases.
        }
    \end{subquestion}

    %--------------------------------------------
    \begin{subquestion}{Increase the frequency of the input $0.5$-V sine wave and record your observations. Interpret and discuss the results.}
        \answer{
            \begin{figure}[H]
                \centering
                \includegraphics[scale=\PicScale,angle=0]{Fig/27.jpeg}
                \caption{The Op-Amp can't follow input in high-frequency inputs.}
            \end{figure}
            Op-amps have limitations in following high-frequency inputs due to several factors.
            \begin{itemize}
                \item Speed limit: Op-amps have a built-in speed limit. They can only change their output so fast.
                \item Delay: There's a tiny delay between when a signal goes in and when it comes out. For really fast signals, this delay becomes a problem.
                \item  Internal capacitors: Op-amps have tiny capacitors inside them. These act like small batteries that need to charge and discharge, which slows things down at high speeds.
                \item Gain drop: Op-amps naturally amplify signals less at higher frequencies. It's like turning down the volume as the pitch gets higher.
                \item Noise: At high frequencies, op-amps start to generate more electrical noise, which can mask the actual signal.
            \end{itemize}
        }
    \end{subquestion}

\end{question}


%----------------------------------------------------------------------------------------
%	QUESTION 4
%----------------------------------------------------------------------------------------

\begin{question}

    \questiontext{Build the circuit shown in Fig. \ref{fig:cir4} using an op-amp non-inverting amplifier module. Create a pair of $\pm 18$ V voltages and connect them to the supply connectors of the module.}

    \begin{figure}[H]
        \centering
        \includegraphics[scale=1.2,angle=0]{Fig/cir4.pdf}
        \caption{Non-inverting amplifier.} \label{fig:cir4}
    \end{figure}

    %--------------------------------------------
    \begin{subquestion}{Apply a $0.5$-V $1$-kHz sine voltage $v_{s}(t)$ to the input of the amplifier. Watch the the output and input voltages of the amplifier simultaneously on the oscilloscope. Calculate the gain of the amplifier.}
        \answer{
            \begin{figure}[H]
                \centering
                \includegraphics[scale=\PicScale,angle=0]{Fig/28.jpeg}
                \caption{The circuit.}
            \end{figure}
            \begin{figure}[H]
                \centering
                \includegraphics[scale=\PicScale,angle=0]{Fig/29.jpeg}
                \caption{oscilloscope's screen for $0.5V$ voltage.}
            \end{figure}
            \begin{figure}[H]
                \centering
                \includegraphics[scale=\PicScale,angle=0]{Fig/30.jpeg}
                \caption{oscilloscope's screen for $5mV$ voltage which used for Calculating Op-Amp's gain.}
            \end{figure}
            We use the data from the second picture because the op amp is not saturated.
            \[
                Gain = \frac{V_{out}}{V_{in}} = \frac{3 \times 100mV}{1.5 \times 5mV} = 40
            \]
        }
    \end{subquestion}

    %--------------------------------------------
    \begin{subquestion}{Measure the input and output resistance of the amplifier module experimentally.}
        \answer{
            For finding $R_{in}$ we connect a $R_s$ after $V_s$ and before $V_1$, we have:
            \[
                V_1 = \frac{R_{in}}{R_{in} + R_s} V_s
            \]
            \begin{figure}[H]
                \centering
                \includegraphics[scale=\PicScale,angle=0]{Fig/44.jpeg}
                \caption{Circuit for $R_{in}$.}
            \end{figure}
            \begin{figure}[H]
                \centering
                \includegraphics[scale=\PicScale,angle=0]{Fig/45.jpeg}
                \caption{oscilloscope's screen for $R_{in}$.}
            \end{figure}
            \begin{figure}[H]
                \centering
                \includegraphics[scale=0.1,angle=0]{Fig/46.jpeg}
                \includegraphics[scale=0.1,angle=0]{Fig/47.jpeg}
                \caption{voltages measured by multimeter for more accurate number.}
            \end{figure}
            \[
                \frac{R_{in}}{R_{in} + 10k} = \frac{0.2913}{0.2920} \Rightarrow R_{in} = 4.1M\Omega
            \]
            For finding $R_{out}$ We connect a resistor $R_L$. At first $R_L \to \infty$ so we find $AV_d$ and then we set $R_L = xk\Omega$, thus:
            \[
                V_2 = AV_d \frac{R_L}{R_L + R_{out}}
            \]
            \begin{figure}[H]
                \centering
                \includegraphics[scale=\PicScale,angle=0]{Fig/48.jpeg}
                \caption{Circuit for $R_{out}$ when $R_L \to \infty$.}
            \end{figure}
            \begin{figure}[H]
                \centering
                \includegraphics[scale=\PicScale,angle=0]{Fig/49.jpeg}
                \caption{oscilloscope's screen for $R_{in}$ when $R_L \to \infty$.}
            \end{figure}
            \begin{figure}[H]
                \centering
                \includegraphics[scale=\PicScale,angle=0]{Fig/50.jpeg}
                \caption{oscilloscope's screen for $R_{in}$ when $R_L = 1k\Omega$.}
            \end{figure}
            \[
                \frac{1k}{1k + R_{out}} = \frac{7}{9} \Rightarrow R_{out} = 0.3k\Omega
            \]
        }
    \end{subquestion}

    %--------------------------------------------
    \begin{subquestion}{Increase the amplitude of the input $1$-kHz sine wave and record your observations. Interpret and discuss the results.}
        \answer{
            \begin{figure}[H]
                \centering
                \includegraphics[scale=0.08,angle=0]{Fig/31.png}
                \includegraphics[scale=0.08,angle=0]{Fig/32.png}
                \includegraphics[scale=0.08,angle=0]{Fig/33.png}
                \caption{oscilloscope's screen for different input voltages.}
            \end{figure}
            As you can see, the op-amp starts to saturate as the amplitude increases.
        }
    \end{subquestion}

    %--------------------------------------------
    \begin{subquestion}{Increase the frequency of the input $0.5$-V sine wave and record your observations. Interpret and discuss the results.}
        \answer{
            \begin{figure}[H]
                \centering
                \includegraphics[scale=\PicScale,angle=0]{Fig/34.jpeg}
                \caption{The Op-Amp can't follow input in high-frequency inputs.}
            \end{figure}
            As explained in experiment 4 section d, op-amp have limitations over high frequency.
        }
    \end{subquestion}

\end{question}

%----------------------------------------------------------------------------------------
%	QUESTION 5
%----------------------------------------------------------------------------------------

\begin{question}

    \questiontext{Cascade an inverting amplifier and a non-inverting amplifier as shown in Fig. \ref{fig:cir5}}.

    \begin{figure}[H]
        \centering
        \includegraphics[scale=1.2,angle=0]{Fig/cir5.pdf}
        \caption{Cascade of two amplifiers.} \label{fig:cir5}
    \end{figure}

    %--------------------------------------------
    \begin{subquestion}{Apply a $100$-mV $1$-kHz sine voltage $v_{s}(t)$ to the input of the cascaded amplifiers. Watch the the output and input voltages of the cascaded amplifiers simultaneously on the oscilloscope. Calculate the overall gain of the cascaded amplifiers.}
        \answer{
            \begin{figure}[H]
                \centering
                \includegraphics[scale=\PicScale,angle=0]{Fig/35.jpeg}
                \caption{The circuit.}
            \end{figure}
            \begin{figure}[H]
                \centering
                \includegraphics[scale=\PicScale,angle=0]{Fig/36.jpeg}
                \caption{oscilloscope's screen.}
            \end{figure}
            \[
                Gain = \frac{V_{out}}{V_{in}} = \frac{2 \times 5V}{1.5 \times 5mV} = 1333.3
            \]
            We also can Calculate Gain using $G_{tot}=G_{inv}G_{nnv}$.
            \[
                Gain_{total} = 26.6 \times 40 = 1064
            \]
        }
    \end{subquestion}

    %--------------------------------------------
    \begin{subquestion}{Swap the order of the amplifiers and repeat the previous part. Is there any difference between the measured gains in the two experiments? Explain.}
        \answer{
            \begin{figure}[H]
                \centering
                \includegraphics[scale=\PicScale,angle=0]{Fig/37.jpeg}
                \caption{The oscilloscope's screen.}
            \end{figure}
            \[
                Gain = \frac{V_{out}}{V_{in}} = \frac{2 \times 5V}{1.5 \times 5mV} = 1333.3
            \]
            By moving these two amplifiers, there was no noticeable change in the gain of the circuit (of course, we may not have noticed the changes due to the accuracy of the oscilloscope), although it would not be strange if there was a difference (more explanation in experiment b of question 6).
        }
    \end{subquestion}

\end{question}


\assignmentSection{Bonus Experiments}

%----------------------------------------------------------------------------------------
%	QUESTION 6
%----------------------------------------------------------------------------------------

\begin{question}

    \questiontext{In a circuit design, we need to cascade an inverting and a non-inverting amplifier to get the overall gain of $G_{tot}=G_{inv}G_{nnv}$. }

    %-------------------------------------------------------------
    \begin{subquestion}{From analytical point of view, is there any difference to change the order of the cascaded amplifiers?}
        \answer{
            Theoretically, the order of multiplication doesn't change the result. So, mathematically:
            \[
                G_{tot} = G_{inv} \times G_{nnv} = G_{nnv} \times G_{inv}
            \]
            This suggests that the order of cascading shouldn't matter for the overall gain. \\
            An inverting amplifier introduces a $180^{\circ}$ phase shift, while a non-inverting amplifier doesn't introduce any phase shift $0^{\circ}$. The total phase shift will be $180^{\circ}$ regardless of the order:
        }
    \end{subquestion}

    %-------------------------------------------------------------
    \begin{subquestion}{From practical point of view, is there any difference to change the order of the cascaded amplifiers? Justify your answer using PSpice simulation.}
        \answer{
            Yes, there are some difference.
            \begin{itemize}
                \item Input impedance: Non-inverting amplifiers typically have higher input impedance than inverting amplifiers. Placing the non-inverting amplifier first in the cascade can provide a higher overall input impedance. This is beneficial because:
                \begin{itemize}
                    \item It reduces loading on the signal source
                    \item It minimizes signal attenuation at the input
                    \item It can improve the overall signal-to-noise ratio
                \end{itemize}
                \item Output impedance: The output impedance of the first stage interacts with the input impedance of the second stage. This interaction can affect:
                \begin{itemize}
                    \item Signal transfer between stages
                    \item Bandwidth of the overall system
                    \item Potential for oscillations or instability
                \end{itemize}
                \item Noise considerations: In a cascade, the noise contribution of the first stage is generally more significant. This is because its noise gets amplified by subsequent stages. Therefore:
                \begin{itemize}
                    \item Placing the lower noise amplifier first can improve the overall noise performance
                    \item This is especially important in low-signal applications where maintaining signal-to-noise ratio is crucial
                \end{itemize}
                \item Bandwidth: Each amplifier stage has its own bandwidth limitations. In a cascade:
                \begin{itemize}
                    \item The overall bandwidth is typically less than that of individual amplifiers
                    \item The order of cascading can affect the final bandwidth, especially if the amplifiers have very different bandwidth characteristics
                    \item Placing the higher bandwidth stage first might help preserve more of the signal's frequency content
                \end{itemize}
                \item Dynamic range and linearity:
                The first stage in a cascade is more susceptible to overload from large input signals.
                \begin{itemize}
                    \item Placing the lower gain stage first can help prevent early saturation
                    \item This can improve the overall linearity of the system
                    \item It's particularly important when dealing with signals that have a wide dynamic range
                \end{itemize}
                \item DC offset: Any DC offset present at the output of the first stage gets amplified by the second stage.
            \end{itemize}
            \begin{figure}[H]
                \centering
                \includegraphics[scale=0.3,angle=0]{Fig/Q6a.png}
                \caption{A simple circuit to check the effect of the order of the amplifiers on the final output.}
            \end{figure}
            \begin{figure}[H]
                \centering
                \includegraphics[scale=0.25,angle=0]{Fig/Q6b.png}
                \caption{The input diagram (which is multiplied by $10$ to be visible on the diagram) and the output of two modes.}
            \end{figure}
            As you can see, there is a phase difference between the two modes and there is a difference between their output voltage.
        }
    \end{subquestion}

\end{question}


%----------------------------------------------------------------------------------------
%	QUESTION 7
%----------------------------------------------------------------------------------------
\begin{question}

    \questiontext{Op-amps usually need a pair of positive and negative DC supply voltages $\pm V_s$.}

    %-------------------------------------------------------------
    \begin{subquestion}{What happens if the absolute values of the supply voltages differ? }
        \answer{
            When the absolute values of these supply voltages differ, it affects the op-amp's performance in several ways:
            \begin{itemize}
                \item Output swing: The maximum output voltage swing will be limited by the smaller of the two supply voltages. For example, if you have $+12V$ and $-5V$ supplies, the output swing will be more restricted in the negative direction.
                \item Offset: An imbalance in supply voltages can introduce an offset voltage at the output, even when the inputs are balanced. This is because the internal circuitry of the op-amp may not be perfectly symmetrical.
                \item Common-mode range: The input common-mode range will shift towards the larger supply voltage. This means the range of input voltages that the op-amp can handle without distortion will be asymmetrical.
                \item Power consumption: The op-amp may consume more power from one supply than the other, which could be an issue in some designs.
                \item Stability: In some cases, significantly unbalanced supply voltages might affect the op-amp's stability, potentially leading to oscillations or other undesired behaviors.
            \end{itemize}
        }
    \end{subquestion}

    %-------------------------------------------------------------
    \begin{subquestion}{Is it possible to use an op-amp with the supply voltages $0$ and $+V_s$? Explain.}
        \answer{
            Yes, This configuration is known as single-supply operation.
            \begin{itemize}
                \item Many modern op-amps are designed to work with a single supply voltage. In this case, the inverting leg is connected to ground $0V$ and the non-inverting leg is connected to $+V_{s}$.
                \item The input common-mode range and output swing will be limited compared to a dual-supply configuration.O(The output can swing from near $0$ to near $+V_{s}$.)
                \item To utilize the full range of the op-amp, input signals often need to be biased to a mid-supply voltage (around $+\frac{V_{s}}{2}$). This creates a "virtual ground" that allows the op-amp to handle both positive and negative variations of the input signal.
            \end{itemize}
            The op-amp cannot produce negative output voltages. Care must be taken to ensure the input doesn't go below ground, which could cause the op-amp to behave unpredictably. \\
            Single-supply op-amps are common in battery-powered devices and systems where generating a negative supply would be inconvenient or inefficient.
        }
    \end{subquestion}

\end{question}


%----------------------------------------------------------------------------------------
%	QUESTION 8
%----------------------------------------------------------------------------------------

\begin{question}

    \questiontext{Return your work report by filling the \LaTeX template of the manual. Include useful and high-quality images to make the report more readable and understandable.}

\end{question}

%----------------------------------------------------------------------------------------

\end{document}
