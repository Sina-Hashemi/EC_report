%%%%%%%%%%%%%%%%%%%%%%%%%%%%%%%%%%%%%%%%%
% Cleese Assignment (For Students)
% LaTeX Template
% Version 2.0 (27/5/2018)
%
% This template originates from:
% http://www.LaTeXTemplates.com
%
% Author:
% Vel (vel@LaTeXTemplates.com)
%
% License:
% CC BY-NC-SA 3.0 (http://creativecommons.org/licenses/by-nc-sa/3.0/)
% 
%%%%%%%%%%%%%%%%%%%%%%%%%%%%%%%%%%%%%%%%%

%----------------------------------------------------------------------------------------
%	PACKAGES AND OTHER DOCUMENT CONFIGURATIONS
%----------------------------------------------------------------------------------------

\documentclass[11pt]{article}
\usepackage{float}

%\usepackage[printwatermark]{xwatermark}
%\newwatermark[allpages,color=gray!50,angle=45,scale=2.5,xpos=-5,ypos=-5]{Mohammad Hadi}

\input{structure.tex} % Include the file specifying the document structure and custom commands

%----------------------------------------------------------------------------------------
%	ASSIGNMENT INFORMATION
%----------------------------------------------------------------------------------------

% Required
\newcommand{\assignmentQuestionName}{Experiment} % The word to be used as a prefix to question numbers; example alternatives: Problem, Exercise
\newcommand{\assignmentClass}{Electrical Circuits Lab (Taught by Mohammad Hadi)\\Manual 2 (Due on DDD.,\ mmm.\ dd,\ yyyy)} % Course (Lecturer)\\Assignment (Due date)
\newcommand{\assignmentTitle}{} % Assignment title or name
\newcommand{\assignmentAuthorName}{Sina Hashemi \& M.Mahdi Shokrzade\\402102668 - 402101985} % Student name\\Student number
%----------------------------------------------------------------------------------------

\newcommand{\PicScale}{0.2}

\begin{document}
\textbf{Function generator is an electronic equipment used to generate different periodic electrical signals. DC power supply is an instrument that gives DC voltage to power a device. Multi-meter is a measuring instrument that can measure various electrical properties. In this experiment, you learn how to work with function generator, DC power supply, and diginal multi-meter.
}
%----------------------------------------------------------------------------------------
%	TITLE PAGE
%----------------------------------------------------------------------------------------

\assignmentSection{Mandatory Experiments}


%----------------------------------------------------------------------------------------
%	QUESTION 1
%----------------------------------------------------------------------------------------

\begin{question}

    \questiontext{Set the controls of the function generator to produce a sine wave of $1$ kHz frequency and $2$ V amplitude.  }

    \begin{subquestion}{Use the oscilloscope to see the wave and fine tune its frequency and amplitude.}
        \answer{
            \begin{figure}[H]
                \begin{center}
                    \includegraphics[scale=\PicScale]{Fig/1.jpeg}
                    \caption{wave on oscilloscope.}
                \end{center}
            \end{figure}
        }
    \end{subquestion}

    \begin{subquestion}{Investigate the impact of attenuation, offset, and duty knobs on the signal.}
        \answer{
            \begin{figure}[H]
                \begin{center}
                    \includegraphics[scale=0.1]{Fig/2.jpeg}
                    \includegraphics[scale=0.1]{Fig/3.jpeg}
                    \caption{attenuation set.}
                \end{center}
            \end{figure}

            \begin{figure}[H]
                \begin{center}
                    \includegraphics[scale=0.1]{Fig/4.jpeg}
                    \includegraphics[scale=0.1]{Fig/5.jpeg}
                    \caption{gave offset.}
                \end{center}
            \end{figure}

            \begin{figure}[H]
                \begin{center}
                    \includegraphics[scale=\PicScale]{Fig/6.jpeg}
                    \caption{changed duty cycle.}
                \end{center}
            \end{figure}
        }
    \end{subquestion}

    \begin{subquestion}{Repeat this part for triangle and square waves.}
        \answer{
            \begin{figure}[H]
                \begin{center}
                    \includegraphics[scale=\PicScale]{Fig/7.jpeg}
                    \caption{triangle wave base form.}
                \end{center}
            \end{figure}

            \begin{figure}[H]
                \begin{center}
                    \includegraphics[scale=0.1]{Fig/8.jpeg}
                    \includegraphics[scale=0.1]{Fig/9.jpeg}
                    \caption{attenuation set.}
                \end{center}
            \end{figure}

            \begin{figure}[H]
                \begin{center}
                    \includegraphics[scale=0.1]{Fig/10.jpeg}
                    \includegraphics[scale=0.1]{Fig/11.jpeg}
                    \caption{gave offset.}
                \end{center}
            \end{figure}

            \begin{figure}[H]
                \begin{center}
                    \includegraphics[scale=\PicScale]{Fig/12.jpeg}
                    \caption{changed duty cycle.}
                \end{center}
            \end{figure}

            \begin{figure}[H]
                \begin{center}
                    \includegraphics[scale=\PicScale]{Fig/13.jpeg}
                    \caption{square wave base form.}
                \end{center}
            \end{figure}

            \begin{figure}[H]
                \begin{center}
                    \includegraphics[scale=0.1]{Fig/14.jpeg}
                    \includegraphics[scale=0.1]{Fig/15.jpeg}
                    \caption{attenuation set.}
                \end{center}
            \end{figure}

            \begin{figure}[H]
                \begin{center}
                    \includegraphics[scale=0.1]{Fig/16.jpeg}
                    \includegraphics[scale=0.1]{Fig/17.jpeg}
                    \caption{gave offset.}
                \end{center}
            \end{figure}

            \begin{figure}[H]
                \begin{center}
                    \includegraphics[scale=\PicScale]{Fig/18.jpeg}
                    \caption{changed duty cycle.}
                \end{center}
            \end{figure}
        }
    \end{subquestion}

\end{question}

%----------------------------------------------------------------------------------------
%	QUESTION 2
%----------------------------------------------------------------------------------------

\begin{question}

    \questiontext{Set the controls of the function generator to produce a sine wave of $50$ Hz frequency and $1$ V amplitude around an offset of $1$ V.}

    \begin{subquestion}{Use the oscilloscope to see the wave and read its frequency and amplitude.}
        \answer{
            \begin{figure}[H]
                \begin{center}
                    \includegraphics[scale=\PicScale]{Fig/19.jpeg}
                    \caption{triangle wave base form.}
                \end{center}
            \end{figure}

            \paragraph*{}
            Frequency: $50$ Hz, Amplitude: $1$ V.
        }
    \end{subquestion}

    \begin{subquestion}{Measure the voltage of the wave using a multi-meter in AC and DC modes. Compare the readings with that of the oscilloscope and interpret the results.}
        \answer{
            \begin{figure}[H]
                \begin{center}
                    \includegraphics[scale=\PicScale]{Fig/20.jpeg}
                    \caption{$V_{AC}$.}
                \end{center}
            \end{figure}

            \begin{figure}[H]
                \begin{center}
                    \includegraphics[scale=\PicScale]{Fig/21.jpeg}
                    \caption{$V_{DC}$.}
                \end{center}
            \end{figure}

            % Ac = 0.73 v Dc = 1.02 v
            \paragraph*{}
            $V_{AC} = 0.73$ V, $V_{DC} = 1.02$ V.
            The DC measure shows the offset and AC measure shows the amplitude of the wave.
            So the amplitude will not be shown in the DC measure and wise versa.

        }
    \end{subquestion}

    \begin{subquestion}{Change the frequency, amplitude, and offset of the wave and check the readings of the multi-meter and oscilloscope.}
        \answer{
            \begin{figure}[H]
                \begin{center}
                    \includegraphics[scale=\PicScale]{Fig/22.jpeg}
                    \caption{frequency increased.}
                \end{center}
            \end{figure}

            \begin{figure}[H]
                \begin{center}
                    \includegraphics[scale=\PicScale]{Fig/23.jpeg}
                    \caption{$V_{AC}$.}
                \end{center}
            \end{figure}
            $V_{DC}$ didn't change.

            \begin{figure}[H]
                \begin{center}
                    \includegraphics[scale=\PicScale]{Fig/24.jpeg}
                    \caption{amplitude changed.}
                \end{center}
            \end{figure}
            \begin{figure}[H]
                \begin{center}
                    \includegraphics[scale=\PicScale]{Fig/25.jpeg}
                    \caption{$V_{AC}$.}
                \end{center}
            \end{figure}

            \begin{figure}[H]
                \begin{center}
                    \includegraphics[scale=\PicScale]{Fig/26.jpeg}
                    \caption{$V_{DC}$.}
                \end{center}
            \end{figure}

            \begin{figure}[H]
                \begin{center}
                    \includegraphics[scale=\PicScale]{Fig/27.jpeg}
                    \caption{offset given.}
                \end{center}
            \end{figure}
            \begin{figure}[H]
                \begin{center}
                    \includegraphics[scale=\PicScale]{Fig/28.jpeg}
                    \caption{$V_{AC}$.}
                \end{center}
            \end{figure}

            \begin{figure}[H]
                \begin{center}
                    \includegraphics[scale=\PicScale]{Fig/29.jpeg}
                    \caption{$V_{DC}$.}
                \end{center}
            \end{figure}
        }
    \end{subquestion}

    \begin{subquestion}{Repeat this part for a square wave of $50\%$ duty cycle with a same frequency, amplitude, and offset as the previous sine wave. Investigate the readings as the duty cycle changes.}
        \answer{
            \begin{figure}[H]
                \begin{center}
                    \includegraphics[scale=\PicScale]{Fig/30.jpeg}
                    \caption{base wave.}
                \end{center}
            \end{figure}
            \begin{figure}[H]
                \begin{center}
                    \includegraphics[scale=\PicScale]{Fig/31.jpeg}
                    \caption{$V_{AC}$.}
                \end{center}
            \end{figure}

            \begin{figure}[H]
                \begin{center}
                    \includegraphics[scale=\PicScale]{Fig/32.jpeg}
                    \caption{$V_{DC}$.}
                \end{center}
            \end{figure}

            \begin{figure}[H]
                \begin{center}
                    \includegraphics[scale=\PicScale]{Fig/33.jpeg}
                    \caption{Duty cycle changed.}
                \end{center}
            \end{figure}
            \begin{figure}[H]
                \begin{center}
                    \includegraphics[scale=\PicScale]{Fig/34.jpeg}
                    \caption{$V_{AC}$.}
                \end{center}
            \end{figure}

            \begin{figure}[H]
                \begin{center}
                    \includegraphics[scale=\PicScale]{Fig/35.jpeg}
                    \caption{$V_{DC}$.}
                \end{center}
            \end{figure}
        }
    \end{subquestion}

\end{question}

%----------------------------------------------------------------------------------------
%	QUESTION 3
%----------------------------------------------------------------------------------------

\begin{question}

    \questiontext{Tune the power supply to generate a $10$ V DC voltage on its master output.}

    \begin{subquestion}{Drive a $10$ k$\Omega$ resistor with the set voltage and measure its current using the multi-meter. Compare the measured and analytical results and justify any difference.}
        \answer{
            For this experiment, the current measuring module
            of our multimeter was not functioning properly.
            As a result, the instructor conducted this experiment for
            all participants.}
    \end{subquestion}

    \begin{subquestion}{Repeat this part for a $1$ k$\Omega$ resistor.}
        \answer{
            \begin{figure}[H]
                \begin{center}
                    \includegraphics[scale=\PicScale]{Fig/36.jpeg}
                    \caption{DC power supply.}
                \end{center}
            \end{figure}
            \begin{figure}[H]
                \begin{center}
                    \includegraphics[scale=\PicScale]{Fig/37.jpeg}
                    \caption{Current of $1$k$\Omega$ resistor.}
                \end{center}
            \end{figure}
        }
    \end{subquestion}

\end{question}


%----------------------------------------------------------------------------------------
%	QUESTION 4
%----------------------------------------------------------------------------------------

\begin{question}

    \questiontext{Set the master output of the DC power supply to $1$ V and its corresponding current limit volume to $0$ A. Then, short circuit the master output using an alligator clip wire and turn the current volume slowly to $0.5$ A. }

    \begin{subquestion}{Discuss the observations.}
        \answer{
            note for my friend. This part doesn't have photo.I
            just have a low quality photo because this part have been done by Hadi. \\ \\

            answer note to my friend: It's ok to put those low quality photos here since it wasn't done by our team.
        }
    \end{subquestion}

    \begin{subquestion}{What do the C.C and C.V LEDs show?}
        \answer{
            \paragraph*{}
            The C.C LED indicates that the power supply is in constant current mode.
            The C.V LED indicates that the power supply is in constant voltage mode.
        }
    \end{subquestion}

    \begin{subquestion}{Can you propose an experiment to measure the resistance of the alligator clip wire?}
        \answer{
            \paragraph*{}
            The resistance of the alligator clip wire can be measured
            by determining the voltage and current flowing through the wire.
            According to Ohm's law, the resistance
            can be calculated using the equation: $R = \frac{V}{I}$
            where $(R)$ is the resistance, $(V)$ is the voltage,
            and $(I)$ is the current. By measuring the voltage
            and current of the wire, we can substitute these values into
            the equation to obtain the resistance.
        }
    \end{subquestion}

\end{question}

%----------------------------------------------------------------------------------------
%	QUESTION 5
%----------------------------------------------------------------------------------------

\begin{question}

    \questiontext{The DC power supply has three operational modes of independent, series, and parallel.}

    \begin{subquestion}{Use the supply to create two independent $2$ V and $3$ V DC voltages and measure them using the multi-meter.}
        \answer{
            \begin{figure}[H]
                \begin{center}
                    \includegraphics[scale=\PicScale]{Fig/38.jpeg}
                    \caption{DC power supply setup.}
                \end{center}
            \end{figure}

            \begin{figure}[H]
                \begin{center}
                    \includegraphics[scale=0.1]{Fig/39.jpeg}
                    \includegraphics[scale=0.1]{Fig/40.jpeg}
                    \caption{Multimeter.}
                \end{center}
            \end{figure}
        }
    \end{subquestion}

    \begin{subquestion}{Use the supply in parallel mode to create a $3$ V DC voltage and measure it using the multi-meter. How is this voltage different from a $3$ V voltage generated from the master output on independent mode?}
        \answer{
            \begin{figure}[H]
                \begin{center}
                    \includegraphics[scale=\PicScale]{Fig/41.jpeg}
                    \caption{DC power supply setup.}
                \end{center}
            \end{figure}

            \begin{figure}[H]
                \begin{center}
                    \includegraphics[scale=\PicScale]{Fig/42.jpeg}
                    \caption{Multimeter for both outputs.}
                \end{center}
            \end{figure}


            \paragraph*{}
            The voltage generated from the master output
            in independent mode is more stable than the voltage
            generated in parallel mode.
            The voltage in parallel mode is less than $3$ V.
        }
    \end{subquestion}

    \begin{subquestion}{Use the supply in series mode to create two $\pm 2$ V DC voltages and measure them using the multi-meter.}
        \answer{
            \begin{figure}[H]
                \begin{center}
                    \includegraphics[scale=\PicScale]{Fig/43.jpeg}
                    \caption{DC power supply setup.}
                \end{center}
            \end{figure}

            \begin{figure}[H]
                \begin{center}
                    \includegraphics[scale=0.1]{Fig/44.jpeg}
                    \includegraphics[scale=0.1]{Fig/45.jpeg}
                    \caption{Multimeter.}
                \end{center}
            \end{figure}
        }
    \end{subquestion}

    \begin{subquestion}{Use the supply to create two $-2$ V and $3$ V DC voltages and measure them using the multi-meter.}
        \answer{
            \begin{figure}[H]
                \begin{center}
                    \includegraphics[scale=\PicScale]{Fig/46.jpeg}
                    \caption{DC power supply setup.}
                \end{center}
            \end{figure}

            \begin{figure}[H]
                \begin{center}
                    \includegraphics[scale=0.1]{Fig/47.jpeg}
                    \includegraphics[scale=0.1]{Fig/48.jpeg}
                    \caption{Multimeter.}
                \end{center}
            \end{figure}
        }
    \end{subquestion}

\end{question}


%----------------------------------------------------------------------------------------
%	QUESTION 6
%----------------------------------------------------------------------------------------

\begin{question}

    \questiontext{Set the selector dial of the multi-meter to resistance mode.}

    \begin{subquestion}{Measure the resistance of a $1$ k$\Omega$ resistor in the most accurate range and compare it with the nominal resistance value.}
        \answer{
            \begin{figure}[H]
                \begin{center}
                    \includegraphics[scale=\PicScale]{Fig/49.jpeg}
                    \caption{Multimeter.}
                \end{center}
            \end{figure}

            \paragraph*{}
            The measured resistance was $0.98$ k$\Omega$.
            This is because the resistance of the resistor is not exactly $1$ k$\Omega$
            and it has errors in its value.

        }
    \end{subquestion}

    \begin{subquestion}{Repeat the previous part for a $100$ k$\Omega$ resistor. What happens if you touch the probes of the multi-meter while measuring the resistance?}
        \answer{
            \begin{figure}[H]
                \begin{center}
                    \includegraphics[scale=\PicScale]{Fig/50.jpeg}
                    \caption{Multimeter.}
                \end{center}
            \end{figure}

            \begin{figure}[H]
                \begin{center}
                    \includegraphics[scale=\PicScale]{Fig/51.jpeg}
                    \caption{touching the probes of the multi-meter while measuring the resistance.}
                \end{center}
            \end{figure}


            \paragraph*{}
            The measured resistance of the resistor itself is $98.01$ k$\Omega$.
            When the probes of the multi-meter are touched together, the resistance is measured as $85.41$ k$\Omega$.
            The difference in resistance between these two measurements is due to the additional resistance
            introduced by the human body when the probes are touched.
            When the probes are touched together,
            the total resistance measured is less than
            the resistance of the resistor itself because the total resistance is equal to the sum of the parallel resistance of the resistor
            and the resistance of the human body (from hand to hand).
            This is because the human body acts as an additional
            path for the current to flow through, resulting in a lower overall resistance.
        }
    \end{subquestion}

    \begin{subquestion}{Connect the probes of the multi-meter to each other and read the displayed resistance. Can you measure the resistance of an alligator clip wire using the multi-meter and its probes? }
        \answer{
            \begin{figure}[H]
                \begin{center}
                    \includegraphics[scale=\PicScale]{Fig/52.jpeg}
                    \caption{Multimeter initial resistor.}
                \end{center}
            \end{figure}

            \begin{figure}[H]
                \begin{center}
                    \includegraphics[scale=\PicScale]{Fig/53.jpeg}
                    \caption{Multimeter measuring the resistance of an alligator clip wire.}
                \end{center}
            \end{figure}

            \begin{figure}[H]
                \begin{center}
                    \includegraphics[scale=\PicScale]{Fig/54.jpeg}
                    \caption{setup for measuring the resistance of an alligator clip wire.}
                \end{center}
            \end{figure}

        }
    \end{subquestion}

\end{question}


%----------------------------------------------------------------------------------------
%	QUESTION 7
%----------------------------------------------------------------------------------------

\begin{question}

    \questiontext{Set the selector dial of the multi-meter to continuity test mode.}

    \begin{subquestion}{Connect the probes of the multi-meter to a $10$ k$\Omega$ resistor and see what happens.}
        \answer{
            \begin{figure}[H]
                \begin{center}
                    \includegraphics[scale=\PicScale]{Fig/55.jpeg}
                    \caption{Multimeter.}
                \end{center}
            \end{figure}


        }
    \end{subquestion}


    \begin{subquestion}{Repeat the previous part for a $1$ k$\Omega$ resistor and for short-circuited probes.}
        \answer{
            \begin{figure}[H]
                \begin{center}
                    \includegraphics[scale=\PicScale]{Fig/56.jpeg}
                    \caption{Multimeter.}
                \end{center}
            \end{figure}

            \paragraph*{}
            The multimeter beeps.

        }
    \end{subquestion}

    \begin{subquestion}{Can you determine the threshold value of the continuity test mode using a simple experiment?}
        \answer{
            \begin{figure}[H]
                \begin{center}
                    \includegraphics[scale=\PicScale]{Fig/57.jpeg}
                    \caption{Approximate threshold value.}
                \end{center}
            \end{figure}


            \paragraph*{}
            % create the experiment
            The threshold value of the continuity test mode can be determined by
            connecting a resistor to the
            multimeter and measuring the resistance in the continuity test mode
            (Note that in this case, whether it indicates continuity (usually a beep sound) or not).
            By increasing the resistance of the resistor, the multimeter will stop beeping.

        }
    \end{subquestion}

\end{question}


\assignmentSection{Bonus Experiments}

%----------------------------------------------------------------------------------------
%	QUESTION 8
%----------------------------------------------------------------------------------------

\begin{question}

    \questiontext{How can the legs of a diode be determined using a multi-meter which has no diode test feature?}

    \answer{

        It can be done by following these steps:

        1. Set the multimeter to the resistance mode. \\
        2. Connect the probes of the multimeter to the diode. \\
        3. In one direction, the resistance will be high, and in the other direction, the resistance will be low. \\
        4. The probe connected to the low resistance side is the anode, and the other probe is the cathode.

    }

\end{question}

%----------------------------------------------------------------------------------------
%	QUESTION 9
%----------------------------------------------------------------------------------------
\begin{question}
    \questiontext{The DT9208 multimeter is used to measure a $5$ V DC voltage. The catalog of the multimeter is available online.
    }

    \begin{subquestion}{Calculate the measurement accuracy if the voltage is measured using the $20$ V range.}
        \answer{
            \paragraph*{}
            The accuracy of the multimeter is $0.5\%$ of the reading plus $2$ digits.
            The voltage is $5$ V, and the range is $20$ V.
            Assuming the multimeter's display resolution in the 20 V range is $0.01$ V (which typically represents one digit),
            the accuracy can be calculated using the following equation:
            \begin{equation*}
                \text{Accuracy} = \text{Reading value} \times \frac{0.5}{100} + 2 \times \text{Least significant digit}
            \end{equation*}
            \begin{equation*}
                \text{Accuracy} = 5 \times \frac{0.5}{100} + 2 \times 0.01 = 0.025 + 0.02 = 0.045 \text{ V}
            \end{equation*}
            Therefore, the accuracy of the measurement is $\pm 0.045$ V.
        }
    \end{subquestion}
    %--------------------------------------------
    \begin{subquestion}{Calculate the measurement accuracy if the voltage is measured using the $200$ V range.}
        \answer{
            \paragraph*{}
            The accuracy of the multimeter is $0.5\%$ of the reading plus $2$ digits.
            The voltage is $5$ V, and the range is $200$ V.
            Assuming the display resolution in the 200 V range is $0.1$ V (one digit),
            the accuracy can be calculated using the following equation:
            \begin{equation*}
                \text{Accuracy} = \text{Reading value} \times \frac{0.5}{100} + 2 \times \text{Least significant digit}
            \end{equation*}
            \begin{equation*}
                \text{Accuracy} = 5 \times \frac{0.5}{100} + 2 \times 0.1 = 0.025 + 0.2 = 0.225 \text{ V}
            \end{equation*}
            Therefore, the accuracy of the measurement is $\pm 0.225$ V.
        }
    \end{subquestion}

    %--------------------------------------------
    \begin{subquestion}{Calculate the measurement accuracy if the voltage is measured using the $1000$ V range.}
        \answer{
            \paragraph*{}
            The accuracy of the multimeter is $0.5\%$ of the reading plus $2$ digits.
            The voltage is $5$ V, and the range is $1000$ V.
            Assuming the display resolution in the 1000 V range is $1$ V (one digit),
            the accuracy can be calculated using the following equation:
            \begin{equation*}
                \text{Accuracy} = \text{Reading value} \times \frac{0.5}{100} + 2 \times \text{Least significant digit}
            \end{equation*}
            \begin{equation*}
                \text{Accuracy} = 5 \times \frac{0.5}{100} + 2 \times 1 = 0.025 + 2 = 2.025 \text{ V}
            \end{equation*}
            Therefore, the accuracy of the measurement is $\pm 2.025$ V.
        }
    \end{subquestion}

\end{question}


%----------------------------------------------------------------------------------------
%	QUESTION 10
%----------------------------------------------------------------------------------------

\begin{question}

    \questiontext{Return your work report by filling the \LaTeX template of the manual. Include useful and high-quality images to make the report more readable and understandable.}

\end{question}

%----------------------------------------------------------------------------------------

\end{document}
