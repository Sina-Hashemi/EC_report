%%%%%%%%%%%%%%%%%%%%%%%%%%%%%%%%%%%%%%%%%
% Cleese Assignment (For Students)
% LaTeX Template
% Version 2.0 (27/5/2018)
%
% This template originates from:
% http://www.LaTeXTemplates.com
%
% Author:
% Vel (vel@LaTeXTemplates.com)
%
% License:
% CC BY-NC-SA 3.0 (http://creativecommons.org/licenses/by-nc-sa/3.0/)
% 
%%%%%%%%%%%%%%%%%%%%%%%%%%%%%%%%%%%%%%%%%

%----------------------------------------------------------------------------------------
%	PACKAGES AND OTHER DOCUMENT CONFIGURATIONS
%----------------------------------------------------------------------------------------

\documentclass[11pt]{article}
\usepackage{float}

%\usepackage[printwatermark]{xwatermark}
%\newwatermark[allpages,color=gray!50,angle=45,scale=2.5,xpos=-5,ypos=-5]{Mohammad Hadi}

\input{structure.tex} % Include the file specifying the document structure and custom commands

%----------------------------------------------------------------------------------------
%	ASSIGNMENT INFORMATION
%----------------------------------------------------------------------------------------

% Required
\newcommand{\assignmentQuestionName}{Experiment} % The word to be used as a prefix to question numbers; example alternatives: Problem, Exercise
\newcommand{\assignmentClass}{Electrical Circuits Lab (Taught by Mohammad Hadi)\\Manual 4 (Due on DDD.,\ mmm.\ dd,\ yyyy)} % Course (Lecturer)\\Assignment (Due date)
\newcommand{\assignmentTitle}{} % Assignment title or name
\newcommand{\assignmentAuthorName}{Sina Hashemi \& M.Mahdi Shokrzade\\402102668 - 402101985} % Student name\\Student number
%----------------------------------------------------------------------------------------

\begin{document}
\textbf{KVL and KCL rules govern every lumped electrical circuit regardless of its elements. In this experiment, you practically verify KCL and KVL rules.
}
%----------------------------------------------------------------------------------------
%	TITLE PAGE
%----------------------------------------------------------------------------------------

\assignmentSection{Mandatory Experiments}

%----------------------------------------------------------------------------------------
%	QUESTION 1
%----------------------------------------------------------------------------------------

\begin{question}

\questiontext{Build the circuit shown in Fig. \ref{fig:cir1} on a breadboard.}

\begin{figure}[H]
\centering
\includegraphics[scale=1.2,angle=0]{Fig/cir1.pdf}
\caption{A resistive circuit.} \label{fig:cir1}
\end{figure}

%--------------------------------------------
\begin{subquestion}{Measure the voltages of the circuit using a multimeter and verify the KVL $v_s=v_1+v_2$.} 
\answer{}
\end{subquestion}

%--------------------------------------------
\begin{subquestion}{Measure the currents of the circuit indirectly using a multimeter and verify the KCL $i_1=i_2+i_3$.} 
\answer{}
\end{subquestion}

%--------------------------------------------
\begin{subquestion}{Reverse the direction of the current $i_2$ and the polarity of the voltage $v_3$ in Fig. \ref{fig:cir1} and repeat the previous parts. } 
\answer{}
\end{subquestion}

%--------------------------------------------
\begin{subquestion}{Change the resistors to $10$ k$\Omega$ and repeat the previous parts. } 
\answer{}
\end{subquestion}

\end{question}

%----------------------------------------------------------------------------------------
%	QUESTION 2
%----------------------------------------------------------------------------------------

\begin{question}

\questiontext{Build the circuit shown in Fig. \ref{fig:cir2} on a breadboard and use two oscilloscopes to show the marked voltages. Use external triggering to synchronize the two oscilloscopes.}

\begin{figure}[H]
\centering
\includegraphics[scale=1.2,angle=0]{Fig/cir2.pdf}
\caption{A sample circuit.} \label{fig:cir2}
\end{figure}


%--------------------------------------------
\begin{subquestion}{Verify the KCL $i_1=i_2+i_4$ graphically using the oscilloscopes.} 
\answer{}
\end{subquestion}

%--------------------------------------------
\begin{subquestion}{Change the square wave to sine wave and repeat the previous part.} 
\answer{}
\end{subquestion}

%--------------------------------------------
\begin{subquestion}{Redo the previous parts when the capacitor is replaced with a diode.} 
\answer{}
\end{subquestion}

\end{question}


\assignmentSection{Bonus Experiments}

%----------------------------------------------------------------------------------------
%	QUESTION 3
%----------------------------------------------------------------------------------------
\begin{question}

\questiontext{Write a MATLAB/Python code that receives the data set $(x_i,y_i), i=1,\cdots,n$ and determines the optimal coefficients of the linear curve $y=ax+b$ that fits the data set with the least square error $\epsilon=\sum_{i=1}^n(y_i-ax_i-b)^2$.
}

\answer{}

\end{question}

%----------------------------------------------------------------------------------------
%	QUESTION 4
%----------------------------------------------------------------------------------------

\begin{question}

\questiontext{Tab. \ref{tab:Q1} includes the measured voltage and current pairs for an unknown LTI resistor. Use linear curve fitting to estimate the characteristic curve of the resistor and its resistance.}

\begin{table}[H]
\centering
\begin{tabular}{l c c c c c c c c c c c c c c c c c c c c}
\toprule
\textbf{Voltage (V)} &
$$02.03$$ &
$$04.12$$ &
$$06.25$$ &
$$08.65$$ &
$$10.00$$ &
$$14.26$$ &
$$16.76$$ &
$$17.56$$ &
$$19.75$$ &
$$21.28$$  \\
\textbf{Current (mA)} &
$$00.97$$ &
$$01.90$$ &
$$03.18$$ &
$$03.94$$ &
$$05.41$$ &
$$05.61$$ &
$$06.66$$ &
$$07.43$$ &
$$08.34$$ &
$$10.73$$ \\
\bottomrule
\toprule
\textbf{Voltage (V)} &
$$26.13$$ &
$$25.70$$ &
$$26.37$$ &
$$32.52$$ &
$$32.27$$ &
$$33.38$$ &
$$36.68$$ &
$$36.40$$ &
$$38.72$$ &
$$47.89$$ \\
\textbf{Current (mA)} &
$$11.17$$ &
$$12.11$$ &
$$12.07$$ &
$$14.98$$ &
$$15.36$$ &
$$15.52$$ &
$$17.04$$ &
$$17.64$$ &
$$17.38$$ &
$$18.95$$ \\
\bottomrule
\end{tabular}
\caption{Measured voltages and currents for an LTI resistor.}\label{tab:Q1}
\end{table}

\answer{}

\end{question}



%----------------------------------------------------------------------------------------
%	QUESTION 5
%----------------------------------------------------------------------------------------

\begin{question}

\questiontext{Return your work report by filling the \LaTeX template of the manual. Include useful and high-quality images to make the report more readable and understandable.}

\end{question}

%----------------------------------------------------------------------------------------

\end{document}
