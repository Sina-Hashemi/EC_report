%%%%%%%%%%%%%%%%%%%%%%%%%%%%%%%%%%%%%%%%%
% Cleese Assignment (For Students)
% LaTeX Template
% Version 2.0 (27/5/2018)
%
% This template originates from:
% http://www.LaTeXTemplates.com
%
% Author:
% Vel (vel@LaTeXTemplates.com)
%
% License:
% CC BY-NC-SA 3.0 (http://creativecommons.org/licenses/by-nc-sa/3.0/)
% 
%%%%%%%%%%%%%%%%%%%%%%%%%%%%%%%%%%%%%%%%%

%----------------------------------------------------------------------------------------
%	PACKAGES AND OTHER DOCUMENT CONFIGURATIONS
%----------------------------------------------------------------------------------------

\documentclass[11pt]{article}
\usepackage{float}

% \usepackage[printwatermark]{xwatermark}
% \newwatermark[allpages,color=gray!50,angle=45,scale=2.5,xpos=-5,ypos=-5]{Mohammad Hadi}

\input{structure.tex} % Include the file specifying the document structure and custom commands

%----------------------------------------------------------------------------------------
%	ASSIGNMENT INFORMATION
%----------------------------------------------------------------------------------------

% Required
\newcommand{\assignmentQuestionName}{Experiment} % The word to be used as a prefix to question numbers; example alternatives: Problem, Exercise
\newcommand{\assignmentClass}{Electrical Circuits Lab (Taught by Mohammad Hadi)\\Manual 5 (Due on DDD.,\ mmm.\ dd,\ yyyy)} % Course (Lecturer)\\Assignment (Due date)
\newcommand{\assignmentTitle}{} % Assignment title or name
\newcommand{\assignmentAuthorName}{Sina Hashemi \& M.Mahdi Shokrzade\\402102668 - 402101985} % Student name\\Student number
%----------------------------------------------------------------------------------------

\begin{document}
\textbf{Network theorems can facilitate circuit analysis and design. In this experiment, you practically verify various network theorems including superposition, Thevenin-Norton equivalency, and maximum power transfer.
}
%----------------------------------------------------------------------------------------
%	TITLE PAGE
%----------------------------------------------------------------------------------------

\assignmentSection{Mandatory Experiments}

%----------------------------------------------------------------------------------------
%	QUESTION 1
%----------------------------------------------------------------------------------------

\begin{question}

\questiontext{Build the circuit shown in Fig. \ref{fig:cir1} on a breadboard.}

\begin{figure}[H]
\centering
\includegraphics[scale=1.2,angle=0]{Fig/cir1.pdf}
\caption{A circuit with two voltage sources.} \label{fig:cir1}
\end{figure}

%--------------------------------------------
\begin{subquestion}{Connect $v_{s1}(t)$ and $v_{s2}(t)$ to a $5$V and $10$V DC voltage source, respectively.} 
\answer{}
\end{subquestion}

%--------------------------------------------
\begin{subquestion}{Measure $v_3$ using a multimeter when $v_{s1}(t)$ is on and $v_{s2}(t)$ is off. Make sure that $v_{s2}(t)$ acts like short circuit when it is off.} 
\answer{}
\end{subquestion}

%--------------------------------------------
\begin{subquestion}{Measure $v_3$ using a multimeter when $v_{s2}(t)$ is on and $v_{s1}(t)$ is off. Make sure that $v_{s1}(t)$ acts like short circuit when it is off.} 
\answer{}
\end{subquestion}

%--------------------------------------------
\begin{subquestion}{Measure $v_3$ using a multimeter when both $v_{s1}(t)$ and $v_{s2}(t)$ are on. Verify the superposition theorem.} 
\answer{}
\end{subquestion}

%--------------------------------------------
\begin{subquestion}{Connect $v_{s1}(t)$ and $v_{s2}(t)$ to a $5$V DC voltage source and a $2$V $1$KHz sine voltage source, respectively. Verify the superposition theorem by suitable measurements using a multimeter. Repeat this part while seeing $v_3$ using an oscilloscope.} 
\answer{}
\end{subquestion}

%--------------------------------------------
\begin{subquestion}{Connect $v_{s1}(t)$ and $v_{s2}(t)$ to a $5$ V and $10$ V DC voltage source, respectively, and replace $R_1$ and $R_2$ with two diodes such that the positive leg of each diode connects to the positive side of the corresponding voltage source. Check if the superposition is held or not. Discuss the results.} 
\answer{}
\end{subquestion}

\end{question}

%----------------------------------------------------------------------------------------
%	QUESTION 2
%----------------------------------------------------------------------------------------

\begin{question}

\questiontext{Build the circuit shown in Fig. \ref{fig:cir2} on a breadboard.}

\begin{figure}[H]
\centering
\includegraphics[scale=1.2,angle=0]{Fig/cir2.pdf}
\caption{A linear circuit.} \label{fig:cir2}
\end{figure}


%--------------------------------------------
\begin{subquestion}{Measure the open circuit voltage $V_{oc}$ of the circuit.} 
\answer{}
\end{subquestion}

%--------------------------------------------
\begin{subquestion}{Connect an $R_L=2.2$ k$\Omega$ resistor to the port AB and measure its voltage $V_L$.} 
\answer{}
\end{subquestion}

%--------------------------------------------
\begin{subquestion}{Can you obtain the corresponding Thevenin equivalent circuit using the measured values of $V_{oc}$ and $V_L$.} 
\answer{}
\end{subquestion}

%--------------------------------------------
\begin{subquestion}{Build the Thevenin equivalent circuit on the breadboard. Connect a same load resistor to both circuits and measure its voltage. Interpret the results.} 
\answer{}
\end{subquestion}

\end{question}


%----------------------------------------------------------------------------------------
%	QUESTION 3
%----------------------------------------------------------------------------------------

\begin{question}

\questiontext{Build the circuit shown in Fig. \ref{fig:cir3} on a breadboard.}

\begin{figure}[H]
\centering
\includegraphics[scale=1.2,angle=0]{Fig/cir3.pdf}
\caption{A simple resistive circuit.} \label{fig:cir3}
\end{figure}


%--------------------------------------------
\begin{subquestion}{Calculate the load resistor $R_L$ drawing the maximum power from the source. Connect it to the port AB and measure the consumed power.} 
\answer{}
\end{subquestion}

%--------------------------------------------
\begin{subquestion}{Connect a $1$ k$\Omega$ and a $2.2$ k$\Omega$ resistor to the port and measure the corresponding consumed powers. Discuss the results and verify the maximum power transfer theorem.} 
\answer{}
\end{subquestion}


\end{question}

\assignmentSection{Bonus Experiments}

%----------------------------------------------------------------------------------------
%	QUESTION 4
%----------------------------------------------------------------------------------------
\begin{question}

\questiontext{A Zener diode is a special type of diode designed to reliably allow current to flow backwards when a certain set reverse voltage, known as the Zener voltage, is reached. The characteristic curve of a typical Zener diode is shown in Fig. \ref{fig:zener}. Explain how a Zener diode can be used as a voltage source. Is there any practical or analytical limitation on a voltage source created by a Zener diode?}

\begin{figure}[H]
\centering
\includegraphics[scale=0.5,angle=0]{Fig/zener.png}
\caption{Typical characteristic of a Zener diode.} \label{fig:zener}
\end{figure}

\answer{}

\end{question}
%----------------------------------------------------------------------------------------
%	QUESTION 5
%----------------------------------------------------------------------------------------

\begin{question}

\questiontext{Consider the typical characteristic curve of a Zener diode.}
%---------------------------------------------------------------------------------------
\begin{subquestion}{Propose a piecewise linear approximation for the characteristic curve using vertical and/or horizontal lines. The forward and (Zener) breakdown voltages should be included in the approximation. }

\answer{}

\end{subquestion}

%----------------------------------------------------------------------------------------
\begin{subquestion}{Use the proposed piecewise linear approximation to suggest a model for the Zener diode. You may use ideal diodes, independent sources, or passive LTI resistors in your model. }

\answer{}

\end{subquestion}

%----------------------------------------------------------------------------------------
\begin{subquestion}{Use PSpice simulation to verify the accuracy of the suggested model for D02CZ10 zener diode with the forward and breakdown voltages around $0.7$ and $-10$ V. }

\answer{}

\end{subquestion}

\end{question}



%----------------------------------------------------------------------------------------
%	QUESTION 6
%----------------------------------------------------------------------------------------

\begin{question}

\questiontext{Return your work report by filling the \LaTeX template of the manual. Include useful and high-quality images to make the report more readable and understandable.}

\end{question}

%----------------------------------------------------------------------------------------

\end{document}
