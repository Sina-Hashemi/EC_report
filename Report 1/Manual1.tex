%%%%%%%%%%%%%%%%%%%%%%%%%%%%%%%%%%%%%%%%%
% Cleese Assignment (For Students)
% LaTeX Template
% Version 2.0 (27/5/2018)
%
% This template originates from:
% http://www.LaTeXTemplates.com
%
% Author:
% Vel (vel@LaTeXTemplates.com)
%
% License:
% CC BY-NC-SA 3.0 (http://creativecommons.org/licenses/by-nc-sa/3.0/)
% 
%%%%%%%%%%%%%%%%%%%%%%%%%%%%%%%%%%%%%%%%%

%----------------------------------------------------------------------------------------
%	PACKAGES AND OTHER DOCUMENT CONFIGURATIONS
%----------------------------------------------------------------------------------------

\documentclass[11pt]{article}
\usepackage{float}

%\usepackage[printwatermark]{xwatermark}
%\newwatermark[allpages,color=gray!50,angle=45,scale=2.5,xpos=-5,ypos=-5]{Mohammad Hadi}

\input{structure.tex} % Include the file specifying the document structure and custom commands

%----------------------------------------------------------------------------------------
%	ASSIGNMENT INFORMATION
%----------------------------------------------------------------------------------------

% Required
\newcommand{\assignmentQuestionName}{Experiment} % The word to be used as a prefix to question numbers; example alternatives: Problem, Exercise
\newcommand{\assignmentClass}{Electrical Circuits Lab (Taught by Mohammad Hadi)\\Manual 1 (Due on DDD.,\ mmm.\ dd,\ yyyy)} % Course (Lecturer)\\Assignment (Due date)
\newcommand{\assignmentTitle}{} % Assignment title or name
\newcommand{\assignmentAuthorName}{Sina Hashemi \& M.Mahdi Shokrzade\\402102668 - 402101985} % Student name\\Student number
%----------------------------------------------------------------------------------------

\newcommand{\PicScale}{0.2}

\begin{document}
\textbf{An oscilloscope is an electronic measuring device which provides a two-dimensional time-domain visual representation of electrical signals. In this experiment, you learn how to use a digital oscilloscope.
}
%----------------------------------------------------------------------------------------
%	TITLE PAGE
%----------------------------------------------------------------------------------------

\assignmentSection{Mandatory Experiments}

%----------------------------------------------------------------------------------------
%	QUESTION 1
%----------------------------------------------------------------------------------------


\begin{question}

    \questiontext{Download the catalog of the lab oscilloscope. Review the catalog and describe the functionality of the major control groups on the oscilloscope including  display group, vertical group, horizontal group, and trigger group.}

    \answer{
        \begin{figure}[H]
            \begin{center}
                \includegraphics[scale=\PicScale]{Fig/1.jpeg}
                \caption{Oscilloscope's front view.}
            \end{center}
        \end{figure}
        \begin{figure}[H]
            \begin{center}
                \includegraphics[scale=0.5]{Fig/30.png}
                \caption{Oscilloscope's front panel buttons .}
            \end{center}
        \end{figure}

        \paragraph*{Display group:}
        Display group includes lcd display and function buttons: \\
        $1$- LCD display which is TFT color, 320 x 234 resolution, wide angle view LCD display.   \\
        $2$- Function keys(F1 (top) to F5 (bottom)): Are used to Activate the functions which
        appear in the left side of the LCD display. \\

        \paragraph*{Vertical group:}
        Vertical group Includes following buttons and controllers: \\
        $1$- Vertical position knob: Is used to move the waveform vertically. \\
        $2$- VOLTS/DIV knob: Is used to select the vertical scale for the signal. \\
        $3$- CH1/CH2 key: Are used to configure the vertical scale and coupling mode for each channel. \\
        $4$- MATH key: Is used to perform math operations.

        \paragraph*{Horizontal group:}
        Horizontal group Includes following buttons and controllers: \\
        $1$- Horizontal position knob: Is used to moves the waveform horizontally. \\
        $2$- Horizontal menu key: Is used to configures the horizontal view. \\
        $3$- TIME/DIV knob: Is used to select the horizontal scale. \\

        \paragraph*{Trigger group:}
        Trigger group Includes following buttons and controllers: \\
        $1$- Trigger level knob: Is used to set the trigger level.Using the trigger level knob moves the trigger point up or down.  \\
        $2$- Single trigger key: Is used to select the single trigger mode \\
        $3$- Trigger force key: Is used to acquire the input signal once regardless of the trigger condition at the time. \\
    }

\end{question}

%----------------------------------------------------------------------------------------
%	QUESTION 2
%----------------------------------------------------------------------------------------

\begin{question}

    \questiontext{Turn on the oscilloscope and set its channels to GND input mode. Use vertical position knobs to move and adjust the ground levels. }

    \answer{

        \begin{figure}[H]
            \begin{center}
                \includegraphics[scale=\PicScale]{Fig/2.jpeg}
                \caption{Channel 2 adjusted.}
            \end{center}
        \end{figure}

        \begin{figure}[H]
            \begin{center}
                \includegraphics[scale=\PicScale]{Fig/3.jpeg}
                \caption{Both channels adjusted.}
            \end{center}
        \end{figure}

    }

\end{question}


%----------------------------------------------------------------------------------------
%	QUESTION 3
%----------------------------------------------------------------------------------------

\begin{question}

    \questiontext{Set the first channel to DC input mode while no signal is fed to its probe. Watch how the horizontal axis is swept. Change the time/div knob and see its impact on the sweep speed. }

    \answer{

        \begin{figure}[H]
            \begin{center}
                \includegraphics[scale=\PicScale]{Fig/4.jpeg}
                \caption{first channel DC mode.}
            \end{center}
        \end{figure}

        \begin{figure}[H]
            \begin{center}
                \includegraphics[scale=\PicScale]{Fig/5.jpeg}
                \caption{time/div set to 100ms.}
            \end{center}
        \end{figure}

        \begin{figure}[H]
            \begin{center}
                \includegraphics[scale=\PicScale]{Fig/6.jpeg}
                \caption{time/div set to 10s.}
            \end{center}
        \end{figure}

    }

\end{question}

%----------------------------------------------------------------------------------------
%	QUESTION 4
%----------------------------------------------------------------------------------------

\begin{question}

    \questiontext{Connect a probe to the oscilloscope and set the volt/div knob to its minimum. Touch the probe tip by your finger. What do see on the oscilloscope screen? Change in input mode to AC, DC, and GND and check the results.}

    \answer{

        \begin{figure}[H]
            \begin{center}
                \includegraphics[scale=\PicScale]{Fig/7.jpeg}
                \caption{input mode set to GND.}
            \end{center}
        \end{figure}

        \begin{figure}[H]
            \begin{center}
                \includegraphics[scale=\PicScale]{Fig/8.jpeg}
                \caption{input mode set to AC.}
            \end{center}
        \end{figure}

        \begin{figure}[H]
            \begin{center}
                \includegraphics[scale=\PicScale]{Fig/9.jpeg}
                \caption{input mode set to DC.}
            \end{center}
        \end{figure}

    }

\end{question}

%----------------------------------------------------------------------------------------
%	QUESTION 5
%----------------------------------------------------------------------------------------

\begin{question}

    \questiontext{Connect a proper probe to the first channel of the oscilloscope and watch the calibrated signal of the built-in frequency generator. Read the frequency and amplitude of the calibrated signal. How can this signal be used for calibration check? What happens when the probe is set to 10X?}

    \answer{

        \begin{figure}[H]
            \begin{center}
                \includegraphics[scale=\PicScale]{Fig/10.jpeg}
                \caption{first channel connected to the oscilloscope calibrated signal
                    generator.Pay attention to the channel error.}
            \end{center}
        \end{figure}

        \begin{figure}[H]
            \begin{center}
                \includegraphics[scale=\PicScale]{Fig/11.jpeg}
                \caption{Probe set to 10X.}
            \end{center}
        \end{figure}

        \paragraph*{}
        The frequency and amplitude of the calibrated signal are $1$ KHz and $1$ V respectively. This signal can be used for calibration check by comparing the signal with the calibrated signal.
        When the probe is set to 10X, the amplitude of the signal is reduced by a factor of 10.

    }

\end{question}

%----------------------------------------------------------------------------------------
%	QUESTION 6
%----------------------------------------------------------------------------------------

\begin{question}

    \questiontext{Set the controls of the function generator to produce a sine wave of $1$ kHz frequency and $2$ V amplitude. Use the oscilloscope to see the signal. Read the frequency and amplitude of the sine wave. Is there any difference between the read and set frequencies? Why?}

    \answer{

        \begin{figure}[H]
            \begin{center}
                \includegraphics[scale=\PicScale]{Fig/12.jpeg}
                \caption{a sine wave of $1$ kHz frequency and $2$ V amplitude.}
            \end{center}
        \end{figure}

        \begin{figure}[H]
            \begin{center}
                \includegraphics[scale=\PicScale]{Fig/13.jpeg}
                \caption{A picture of the function generator.}
            \end{center}
        \end{figure}

        \paragraph*{}
        The read frequency is $1.004$ KHz and the read amplitude is $2$ V. The difference between the
        read and set frequencies is due to the errors that might occur in the function generator,
        during the signal transmission or in the oscilloscope.
        So it's hard to set the frequency and amplitude of the signal exactly the same as wanted.
    }

\end{question}

%----------------------------------------------------------------------------------------
%	QUESTION 7
%----------------------------------------------------------------------------------------

\begin{question}

    \questiontext{Add a DC offset to the sine wave in the previous part and check the corresponding signal
        on the oscilloscope screen in different input modes of DC, AC, and GND.}

    \answer{

        \begin{figure}[H]
            \begin{center}
                \includegraphics[scale=\PicScale]{Fig/14.jpeg}
                \caption{input mode set to DC.}
            \end{center}
        \end{figure}

        \begin{figure}[H]
            \begin{center}
                \includegraphics[scale=\PicScale]{Fig/15.jpeg}
                \caption{input mode set to AC.}
            \end{center}
        \end{figure}

        \begin{figure}[H]
            \begin{center}
                \includegraphics[scale=\PicScale]{Fig/16.jpeg}
                \caption{input mode set to GND.}
            \end{center}
        \end{figure}

        \begin{figure}[H]
            \begin{center}
                \includegraphics[scale=\PicScale]{Fig/17.jpeg}
                \caption{Function generator offset changed.}
            \end{center}
        \end{figure}

    }

\end{question}

%----------------------------------------------------------------------------------------
%	QUESTION 8
%----------------------------------------------------------------------------------------

\begin{question}

    \questiontext{Feed the two channels of the oscilloscope by a same $1$ KHz
        sine wave and investigate the functionality of the math operations such as Add and Inv.}

    \answer{

        \begin{figure}[H]
            \begin{center}
                \includegraphics[scale=\PicScale]{Fig/18.jpeg}
                \caption{Both channels are fed with a single function generator.
                    Pay attention to the first channel zero error.}
            \end{center}
        \end{figure}

        \begin{figure}[H]
            \begin{center}
                \includegraphics[scale=\PicScale]{Fig/19.jpeg}
                \caption{Add math function.}
            \end{center}
        \end{figure}

        \begin{figure}[H]
            \begin{center}
                \includegraphics[scale=\PicScale]{Fig/20.jpeg}
                \caption{Inv math function.}
            \end{center}
        \end{figure}

    }

\end{question}

%----------------------------------------------------------------------------------------
%	QUESTION 9
%----------------------------------------------------------------------------------------

\begin{question}

    \questiontext{Repeat the previous part with two $1$ KHz sine waves produced from
        two different function generators. Explain your observations.}

    \answer{

        \begin{figure}[H]
            \begin{center}
                \includegraphics[scale=0.1]{Fig/21.png}
                \includegraphics[scale=0.1]{Fig/22.png}
                \includegraphics[scale=0.1]{Fig/23.png}
                \caption{Two $1$ KHz sine waves produced from two different function generators.}
            \end{center}
        \end{figure}

        \begin{figure}[H]
            \begin{center}
                \includegraphics[scale=0.1]{Fig/24.png}
                \includegraphics[scale=0.1]{Fig/25.png}
                \includegraphics[scale=0.1]{Fig/26.png}
                \caption{Two sine waves gathered with each other.}
            \end{center}
        \end{figure}

        \begin{figure}[H]
            \begin{center}
                \includegraphics[scale=0.1]{Fig/27.png}
                \includegraphics[scale=0.1]{Fig/28.png}
                \includegraphics[scale=0.1]{Fig/29.png}
                \caption{Two sine waves subtracted from each other.}
            \end{center}
        \end{figure}

        \paragraph*{}
        The observed phenomenon in the oscilloscope display can
        be explained as follows: The first wave appears stationary,
        while the second wave appears to be moving.
        This is due to the fact that the oscilloscope determines
        its time reference based on one of the input waves.
        In this case, the two input waves are generated by separate function
        generators, and due to testing errors, their frequencies are
        slightly different. As a result, the second wave appears to be
        shifting in time relative to the first wave on the
        oscilloscope display.

    }

\end{question}

%----------------------------------------------------------------------------------------
%	QUESTION 10
%----------------------------------------------------------------------------------------

\begin{question}

    \questiontext{Use the oscilloscope to watch and measure the potential difference
        between nodes A and B in the circuit of Fig. \ref{fig:cir1}, where the sinusoidal
        source has a frequency of $1$ kHz and a peak to peak amplitude of $5$ V. Can you measure
        the desired voltage simply using a single probe? If no, describe the problem and propose a solution.
    }

    \begin{figure}[H]
        \centering
        \includegraphics[scale=0.8,angle=0]{Fig/cir1.pdf}
        \caption{A simple resistive circuit.} \label{fig:cir1}
    \end{figure}

    \answer{

        \paragraph*{}
        We connected the ground base of both input channels of the oscilloscope
        to the ground base of the function generator.
        Additionally, we connected the probe of the
        first channel to point A and the probe of the second channel
        to point B. By utilizing the subtraction option of the
        oscilloscope, we subtracted these two waveforms to obtain
        the potential difference between the two ends of the resistance.

        \begin{figure}[H]
            \begin{center}
                \includegraphics[scale=\PicScale]{Fig/30.jpeg}
                \caption{A simple resistive circuit.}
            \end{center}
        \end{figure}

        \begin{figure}[H]
            \begin{center}
                \includegraphics[scale=\PicScale]{Fig/31.jpeg}
                \caption{oscilloscope's picture.}
            \end{center}
        \end{figure}

        \begin{figure}[H]
            \begin{center}
                \includegraphics[scale=\PicScale]{Fig/32.jpeg}
                \caption{Function generator's setting.}
            \end{center}
        \end{figure}

        \paragraph*{}
        By interchanging the positive and negative terminals
        of the function generator, and connecting the ground terminal
        of the oscilloscope probe to the negative terminal of the function
        generator (which corresponds to one end of the resistor),
        and connecting the positive terminal of the oscilloscope to the
        other end of the resistor, we can accurately measure the potential
        difference across the resistor. This configuration allows us to obtain
        the desired voltage measurement using a single probe.

        \begin{figure}[H]
            \begin{center}
                \includegraphics[scale=\PicScale]{Fig/33.jpeg}
                \caption{Circuit related to resistance voltage measurement with one probe.}
            \end{center}
        \end{figure}

        \begin{figure}[H]
            \begin{center}
                \includegraphics[scale=\PicScale]{Fig/34.jpeg}
                \caption{Measuring resistance voltage with one probe.}
            \end{center}
        \end{figure}
    }

\end{question}

%----------------------------------------------------------------------------------------
%	QUESTION 11
%----------------------------------------------------------------------------------------

\begin{question}

    \questiontext{Set the controls of the function generator to
        produce a sine wave of $70$ Hz frequency
        and $1$ V amplitude. Connect the sine wave to
        the first channel of the oscilloscope and see it for various
        triggering sources. Discuss the observations.}

    \answer{

        \begin{figure}[H]
            \begin{center}
                \includegraphics[scale=\PicScale]{Fig/35.jpeg}
                \caption{Trigger source set to CH1.}
            \end{center}
        \end{figure}

        \begin{figure}[H]
            \begin{center}
                \includegraphics[scale=0.1]{Fig/36.jpeg}
                \includegraphics[scale=0.1]{Fig/37.jpeg}
                \includegraphics[scale=0.1]{Fig/38.jpeg}
                \caption{Trigger source set to CH2.}
            \end{center}
        \end{figure}

        \begin{figure}[H]
            \begin{center}
                \includegraphics[scale=0.1]{Fig/39.jpeg}
                \includegraphics[scale=0.1]{Fig/40.jpeg}
                \includegraphics[scale=0.1]{Fig/41.jpeg}
                \caption{Trigger source set to External.}
            \end{center}
        \end{figure}

        \begin{figure}[H]
            \begin{center}
                \includegraphics[scale=0.1]{Fig/42.jpeg}
                \includegraphics[scale=0.1]{Fig/43.jpeg}
                \includegraphics[scale=0.1]{Fig/44.jpeg}
                \caption{Trigger source set to Line.}
            \end{center}
        \end{figure}

        \paragraph*{} Oscilloscope has various triggering sources. CH1, CH2, External and Line.

        \paragraph*{CH1 and CH2 sources:} In these two cases, the oscilloscope sets the time origin based on channel 1 and channel 2,
        respectively. Therefore, in the first state, the wave is stationary, but in the second state, the wave moves.

        \paragraph*{External source:} In this case, the oscilloscope sets the time origin based on the
        input given to the third base (external base).

        \paragraph*{Line source:} In this case, the oscilloscope sets the time origin based on the city's electricity frequency.
    }

\end{question}

%----------------------------------------------------------------------------------------
%	QUESTION 12
%----------------------------------------------------------------------------------------

\begin{question}

    \questiontext{Change the trigger level and trigger slope in the previous part and analyze your observations.}

    \answer{
        \begin{figure}[H]
            \begin{center}
                \includegraphics[scale=\PicScale]{Fig/45.jpeg}
                \caption{Default mode.}
            \end{center}
        \end{figure}

        \begin{figure}[H]
            \begin{center}
                \includegraphics[scale=0.1]{Fig/46.jpeg}
                \includegraphics[scale=0.1]{Fig/47.jpeg}
                \includegraphics[scale=0.1]{Fig/48.jpeg}
                \caption{Trigger level changed to +v.}
            \end{center}
        \end{figure}

        \begin{figure}[H]
            \begin{center}
                \includegraphics[scale=0.1]{Fig/49.jpeg}
                \includegraphics[scale=0.1]{Fig/50.jpeg}
                \includegraphics[scale=0.1]{Fig/51.jpeg}
                \caption{Trigger level changed to -v.}
            \end{center}
        \end{figure}

        \begin{figure}[H]
            \begin{center}
                \includegraphics[scale=\PicScale]{Fig/52.jpeg}
                \caption{Trigger slope changed.}
            \end{center}
        \end{figure}

        \begin{figure}[H]
            \begin{center}
                \includegraphics[scale=0.1]{Fig/53.jpeg}
                \includegraphics[scale=0.1]{Fig/54.jpeg}
                \includegraphics[scale=0.1]{Fig/55.jpeg}
                \caption{Trigger level changed to +v.}
            \end{center}
        \end{figure}

        \begin{figure}[H]
            \begin{center}
                \includegraphics[scale=0.1]{Fig/56.jpeg}
                \includegraphics[scale=0.1]{Fig/57.jpeg}
                \includegraphics[scale=0.1]{Fig/58.jpeg}
                \caption{Trigger level changed to -v.}
            \end{center}
        \end{figure}

        \paragraph*{} By changing the level trigger, if it goes out of the wave range, it causes the wave to move on the screen.

        \paragraph*{} By changing the trigger slope, the wave is projected according to the horizontal line.
    }


\end{question}

%----------------------------------------------------------------------------------------
%	QUESTION 13
%----------------------------------------------------------------------------------------

\begin{question}

    \questiontext{Feed two sine waves with differences frequencies to two channels of the scope and see
        the corresponding Lissajous curve on the XY mode. Characterize the Lissajous curve. Is there any relationship
        between the Lissajous curve and the  frequencies of the sine waves? }

    \answer{
        \begin{figure}[H]
            \begin{center}
                \includegraphics[scale=0.1]{Fig/59.jpeg}
                \includegraphics[scale=0.1]{Fig/60.jpeg}
                \includegraphics[scale=0.1]{Fig/61.jpeg}
                \caption{First Function generator set to $100Hz$ and Second one set to $100Hz$.}
            \end{center}
        \end{figure}

        \begin{figure}[H]
            \begin{center}
                \includegraphics[scale=0.1]{Fig/62.jpeg}
                \includegraphics[scale=0.1]{Fig/63.jpeg}
                \includegraphics[scale=0.1]{Fig/64.jpeg}
                \caption{First Function generator set to $200Hz$ and Second one set to $100Hz$.}
            \end{center}
        \end{figure}

        \begin{figure}[H]
            \begin{center}
                \includegraphics[scale=0.1]{Fig/65.jpeg}
                \includegraphics[scale=0.1]{Fig/66.jpeg}
                \includegraphics[scale=0.1]{Fig/67.jpeg}
                \caption{First Function generator set to $70Hz$ and Second one set to $100Hz$.}
            \end{center}
        \end{figure}

        \begin{figure}[H]
            \begin{center}
                \includegraphics[scale=0.1]{Fig/68.jpeg}
                \includegraphics[scale=0.1]{Fig/69.jpeg}
                \includegraphics[scale=0.1]{Fig/70.jpeg}
                \caption{First Function generator set to $34Hz$ and Second one set to $100Hz$.}
            \end{center}
        \end{figure}

        \begin{figure}[H]
            \begin{center}
                \includegraphics[scale=0.1]{Fig/71.jpeg}
                \includegraphics[scale=0.1]{Fig/72.jpeg}
                \includegraphics[scale=0.1]{Fig/73.jpeg}
                \caption{First Function generator set to $200 Hz$ and Second one set to $100Hz$.}
            \end{center}
        \end{figure}

        \[
            \frac{\text{Number of horizontal peaks}}{\text{Number of vertical peaks}} = \frac{\text{The frequency of the second generator}}
            {\text{The frequency of the first generator}}
        \]
    }

\end{question}

%----------------------------------------------------------------------------------------
%	QUESTION 14
%----------------------------------------------------------------------------------------

\begin{question}

    \questiontext{Build the circuit of Fig. \ref{fig:cir2} on the breadboard and see the labeled voltages in
        the XY mode on the oscilloscope screen. Sweep the frequency and amplitudes of the input sine wave and observe
        the results. Can you interpret the results?
    }
    \begin{figure}[H]
        \centering
        \includegraphics[scale=0.8,angle=0]{Fig/cir2.pdf}
        \caption{An RC circuit.} \label{fig:cir2}
    \end{figure}

    \answer{
        \begin{figure}[H]
            \begin{center}
                \includegraphics[scale=0.1]{Fig/74.jpeg}
                \includegraphics[scale=0.1]{Fig/75.jpeg}
                \includegraphics[scale=0.1]{Fig/76.jpeg}
                \includegraphics[scale=0.15]{Fig/77.jpeg}
                \includegraphics[scale=0.15]{Fig/78.jpeg}
            \end{center}
        \end{figure}

        \paragraph*{}
        The circuit consists of a resistor and a capacitor connected in series. The input
        signal is applied to the circuit, and the output signal is measured across the
        capacitor. The output signal is then fed to the oscilloscope, and the XY mode is activated.
        The input signal is a sine wave with a frequency of $1$ KHz and an amplitude of $1$ V.
        The amplitude of the input signal is swept from $0.1$ V to $1$ V,
        and the frequency is swept from $1$ KHz to $10$ KHz. The results are shown in the figures above.
        The results show that the output signal is an oval which is really close to a circle.
        The radius of the circle is proportional to the amplitude of the input signal,
        and the frequency of the input signal determines the speed of the circle.

        \paragraph*{}
        From the circuit, we can represent the input and output voltages as follows:
        \begin{equation*}
            V_i = A \cos(\omega t)
        \end{equation*}
        \begin{equation*}
            V_o = B \cos(\omega t + \theta)
        \end{equation*}
        where $V_i$ is the input voltage, $V_o$ is the output voltage across the capacitor,
        $A$ and $B$ are amplitudes, $\omega$ is the angular frequency, and $\theta$ is the phase difference.

        \paragraph*{}
        When visualizing the voltages on the oscilloscope in XY mode, the resulting plot
        represents a Lissajous figure, given by the equation:
        \begin{equation*}
            \left(\frac{V_x}{A}\right)^2 + \left(\frac{V_y}{B}\right)^2 = 1
        \end{equation*}
        where $V_x$ and $V_y$ are the oscilloscope's X and Y inputs respectively.
        This equation describes an ellipse, and for certain phase relationships, it can form a circle.

        \paragraph*{}
        If we assume specific conditions, such as $\theta = \frac{\pi}{2}$, the equations transform into:
        \begin{equation*}
            V_i = A \sin(\omega t)
        \end{equation*}
        \begin{equation*}
            V_o = A \cos(\omega t)
        \end{equation*}
        This yields:
        \begin{equation*}
            V_i^2 + V_o^2 = A^2
        \end{equation*}
        representing a perfect circle with radius $A$, where $A$ is the amplitude of the input signal.

        \paragraph*{}
        Therefore, the observed output on the oscilloscope is an elliptical pattern that
        approaches a circular shape as the phase difference between the input and output signals
        approaches $\frac{\pi}{2}$. The radius of the observed pattern is directly proportional to the
        amplitude of the input signal. By sweeping the frequency, the speed at which the pattern moves changes,
        demonstrating the frequency dependence of the capacitor's impedance.
    }

\end{question}


\assignmentSection{Bonus Experiments}

%----------------------------------------------------------------------------------------
%	QUESTION 15
%----------------------------------------------------------------------------------------

\begin{question}

    \questiontext{Consider the block diagram of a typical analog oscilloscope shown in Fig.
        \ref{fig:Q1} and explain how an analog oscilloscope works. How does an analog oscilloscope
        differ from its digital counterpart? }

    \begin{figure}[H]
        \centering
        \includegraphics[scale=1,angle=0]{Fig/Q1.png}
        \caption{Block diagram of an analog oscilloscope.} \label{fig:Q1}
    \end{figure}

    \answer{

        \paragraph*{}
        The block diagram of an analog oscilloscope consists of the following components:
        \begin{itemize}
            \item Vertical amplifier: Amplifies the input signal.
            \item Time base generator: Generates the sweep signal.
            \item Trigger circuit: Synchronizes the sweep signal with the input signal.
            \item Cathode ray tube (CRT): Displays the input signal.
        \end{itemize}

        \paragraph*{}
        The input signal is amplified by the vertical amplifier and fed to the vertical deflection plates of the CRT.
        The time base generator generates a sweep signal that is fed to the horizontal deflection plates of the CRT.
        The trigger circuit synchronizes the sweep signal with the input signal to ensure that the input signal is displayed correctly on the screen.

        \paragraph*{}
        The main difference between an analog oscilloscope and a digital oscilloscope is the display technology.
        An analog oscilloscope uses a CRT to display the input signal, while a digital oscilloscope uses an LCD or LED display.
        Digital oscilloscopes offer more features and better accuracy than analog oscilloscopes, but they are more expensive.

    }

\end{question}

%----------------------------------------------------------------------------------------
%	QUESTION 16
%----------------------------------------------------------------------------------------

\begin{question}

    \questiontext{What might lead to a distorted representation of the oscilloscope calibration signal on the screen? What do you offer to resolve the problem?
    }
    \answer{
        \paragraph*{}
        There are several factors that can lead to a distorted representation of the oscilloscope calibration signal on the screen. Some of these factors include:

        \begin{itemize}
            \item Incorrect settings of the oscilloscope: Ensure that the
                  oscilloscope settings, such as timebase, voltage scale, and trigger level,
                  are properly configured for the input signal.
            \item Poor quality of the input signal: Check the input signal source
                  and make sure it is clean and free from noise or interference.
            \item Incorrect connection of the probe: Ensure that the probe
                  is properly connected to the oscilloscope and the circuit under test. Check for loose connections or damaged cables.
            \item Interference from other electronic devices:
                  Keep the oscilloscope away from other
                  electronic devices that may introduce electromagnetic interference.
            \item Incorrect calibration of the oscilloscope: Perform
                  a calibration procedure on the oscilloscope to ensure accurate measurements.
            \item Incorrect grounding of the oscilloscope: Ensure that the oscilloscope
                  is properly grounded to avoid ground loops and unwanted noise.
            \item Incorrect triggering settings: Adjust the triggering settings
                  of the oscilloscope to properly capture the desired waveform.
        \end{itemize}

        \paragraph*{}
        To resolve the problem, you can take the following steps:

        \begin{itemize}
            \item Check the settings of the oscilloscope and adjust them as needed to match the input signal.
            \item Verify the quality of the input signal and eliminate any noise or interference sources.
            \item Double-check the connection of the probe and ensure it is securely attached to
                  the oscilloscope and the circuit.
            \item Minimize the proximity of the oscilloscope to other electronic devices to reduce interference.
            \item Perform a calibration procedure on the oscilloscope to ensure accurate measurements.
            \item Ensure that the oscilloscope is properly grounded to avoid ground loops and unwanted noise.
            \item Adjust the triggering settings of the oscilloscope to capture the desired waveform accurately.
        \end{itemize}

        By following these steps, you can troubleshoot and resolve
        any issues that may lead to a distorted representation
        of the oscilloscope calibration signal on the screen.
    }

\end{question}

%----------------------------------------------------------------------------------------
%	QUESTION 17
%----------------------------------------------------------------------------------------

\begin{question}

    \questiontext{Return your work report by filling the \LaTeX template of the manual.
        Include useful and high-quality images to make the report more readable and understandable.}

\end{question}

%----------------------------------------------------------------------------------------

\end{document}
