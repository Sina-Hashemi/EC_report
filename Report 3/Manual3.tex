%%%%%%%%%%%%%%%%%%%%%%%%%%%%%%%%%%%%%%%%%
% Cleese Assignment (For Students)
% LaTeX Template
% Version 2.0 (27/5/2018)
%
% This template originates from:
% http://www.LaTeXTemplates.com
%
% Author:
% Vel (vel@LaTeXTemplates.com)
%
% License:
% CC BY-NC-SA 3.0 (http://creativecommons.org/licenses/by-nc-sa/3.0/)
% 
%%%%%%%%%%%%%%%%%%%%%%%%%%%%%%%%%%%%%%%%%

%----------------------------------------------------------------------------------------
%	PACKAGES AND OTHER DOCUMENT CONFIGURATIONS
%----------------------------------------------------------------------------------------

\documentclass[11pt]{article}
\usepackage{float}

%\usepackage[printwatermark]{xwatermark}
%\newwatermark[allpages,color=gray!50,angle=45,scale=2.5,xpos=-5,ypos=-5]{Mohammad Hadi}

\input{structure.tex} % Include the file specifying the document structure and custom commands

%----------------------------------------------------------------------------------------
%	ASSIGNMENT INFORMATION
%----------------------------------------------------------------------------------------

% Required
\newcommand{\assignmentQuestionName}{Experiment} % The word to be used as a prefix to question numbers; example alternatives: Problem, Exercise
\newcommand{\assignmentClass}{Electrical Circuits Lab (Taught by Mohammad Hadi)\\Manual 3 (Due on DDD.,\ mmm.\ dd,\ yyyy)} % Course (Lecturer)\\Assignment (Due date)
\newcommand{\assignmentTitle}{} % Assignment title or name
\newcommand{\assignmentAuthorName}{Student Names\\Student Numbers} % Student name\\Student number
%----------------------------------------------------------------------------------------

\begin{document}
\textbf{Real circuit elements are approximate physical implementation of ideal circuit elements. In this experiment, you become familiar with the main limitations of the conventional real circuit elements such as resistors, capacitors, and diodes.
}
%----------------------------------------------------------------------------------------
%	TITLE PAGE
%----------------------------------------------------------------------------------------

\assignmentSection{Mandatory Experiments}

%----------------------------------------------------------------------------------------
%	QUESTION 1
%----------------------------------------------------------------------------------------

\begin{question}

    \questiontext{Consider the circuit shown in Fig. \ref{fig:cir1} }

    \begin{figure}[H]
        \centering
        \includegraphics[scale=1.2,angle=0]{Fig/cir1.pdf}
        \caption{A voltage divider circuit.} \label{fig:cir1}
    \end{figure}

    %--------------------------------------------
    \begin{subquestion}{Build the circuit on a breadboard. Can you find a $9$ k$\Omega$ resistor in the stackable element storage box? }
        \answer{
            %  test comment to see if I have access to project or no!
        }
    \end{subquestion}

    %--------------------------------------------
    \begin{subquestion}{Which resistor is a suitable replacement for a $9$ k$\Omega$ resistor? Pick that resistor and read its color code.}
        \answer{}
    \end{subquestion}

    %--------------------------------------------
    \begin{subquestion}{Measure the voltage across each resistor. Is there any difference between the measured and analytical values?}
        \answer{}
    \end{subquestion}

\end{question}


%----------------------------------------------------------------------------------------
%	QUESTION 2
%----------------------------------------------------------------------------------------

\begin{question}

    \questiontext{Each real circuit element has a nominal value within a tolerance band.}

    %--------------------------------------------
    \begin{subquestion}{Pick up ten E12 $1$ k$\Omega$ resistors and measure their resistance. Calculate the maximum, minimum, mean, variance, and standard deviation of the measured values and interpret them considering the nominal and tolerance values. }
        \answer{}
    \end{subquestion}

    %--------------------------------------------
    \begin{subquestion}{Repeat the previous part for ten E12 $10$ $\mu$F capacitors. }
        \answer{}
    \end{subquestion}

\end{question}

%----------------------------------------------------------------------------------------
%	QUESTION 3
%----------------------------------------------------------------------------------------

\begin{question}

    \questiontext{\underline{This experiment should be done by the lab supervisor}. Dependency of the value of a real circuit element on temperature is characterized by a specified temperature coefficient. }

    %--------------------------------------------
    \begin{subquestion}{Measure the resistance of a carbon resistor in room temperature. Redo the measurement when the resistor is heated by a soldering iron and when is cooled by a cooling spray. }
        \answer{}
    \end{subquestion}

    %--------------------------------------------
    \begin{subquestion}{Repeat the previous part for a metal film resistor, a ceramic capacitor, and a multi-layer capacitor. }
        \answer{}
    \end{subquestion}

\end{question}

%----------------------------------------------------------------------------------------
%	QUESTION 4
%----------------------------------------------------------------------------------------

\begin{question}

    \questiontext{\underline{This experiment should be done by the lab supervisor}. Each resistor has a specific maximum power rating. Connect a $0.25$ W $10$ $\Omega$ resistor to a DC power supply using an alligator clip wire. Sweep the voltage from $0$ to $5$ V and observe the results.}

    \answer{}

\end{question}

%----------------------------------------------------------------------------------------
%	QUESTION 4
%----------------------------------------------------------------------------------------

\begin{question}

    \questiontext{\underline{This experiment should be done by the lab supervisor}. Each capacitor has a specific maximum voltage. Further, a capacitor may have a  specific voltage polarity. }

    %--------------------------------------------
    \begin{subquestion}{Connect a $16$ V aluminum electrolytic capacitor directly to a $30$ V DC voltage and observe the results.}
        \answer{}
    \end{subquestion}

    %--------------------------------------------
    \begin{subquestion}{Connect a $16$ V aluminum electrolytic capacitor inversely to a $16$ V DC voltage and observe the results.}
        \answer{}
    \end{subquestion}

\end{question}

%----------------------------------------------------------------------------------------
%	QUESTION 1
%----------------------------------------------------------------------------------------

\begin{question}

    \questiontext{Build the circuit shown in Fig. \ref{fig:cir2} on a breadboard, where $r=1$ k$\Omega$ denotes a resistor allowing to measure the current indirectly.}

    \begin{figure}[H]
        \centering
        \includegraphics[scale=1.2,angle=0]{Fig/cir2.pdf}
        \caption{A test circuit for extracting the characteristic curve of a resistive element using multimeter.} \label{fig:cir2}
    \end{figure}

    %--------------------------------------------
    \begin{subquestion}{Replace $D$ with a $1$ k$\Omega$ resistor. Sweep the DC voltage over negative and positive ranges and record the voltage and current of the resistor using a multimeter. Use the recorded data to plot the characteristic curve of the resistor.}
        \answer{}
    \end{subquestion}

    %--------------------------------------------
    \begin{subquestion}{Repeat the previous part for a diode.}
        \answer{}
    \end{subquestion}


\end{question}


%----------------------------------------------------------------------------------------
%	QUESTION 1
%----------------------------------------------------------------------------------------

\begin{question}

    \questiontext{The experimental setup of Fig. \ref{fig:cir3} is used to display the Lissajous curve constructed by $v_x(t)$ and
        $v_y(t)$, where $D$ is a basic circuit element and $r=1$ k$\Omega$. }

    \begin{figure}[H]
        \centering
        \includegraphics[scale=1.2,angle=0]{Fig/cir3.pdf}
        \caption{A test circuit for extracting the characteristic curve of a resistive element using oscilloscope.} \label{fig:cir3}
    \end{figure}

    %--------------------------------------------
    \begin{subquestion}{How can the displayed Lissajous curve be used
            to obtain the characteristic curve of a resistive element?}
        \answer{}
    \end{subquestion}

    %--------------------------------------------
    \begin{subquestion}{Can the characteristic curve be extracted using the circuit of Fig. \ref{fig:cir2} and an oscilloscope? }
        \answer{}
    \end{subquestion}

    %--------------------------------------------
    \begin{subquestion}{Replace $D$ with a $1$ k$\Omega$ resistor and see the corresponding characteristic curve on the oscilloscope screen.}
        \answer{}
    \end{subquestion}

    %--------------------------------------------
    \begin{subquestion}{Repeat the previous part for a diode.}
        \answer{}
    \end{subquestion}


\end{question}

%----------------------------------------------------------------------------------------
%	QUESTION 1
%----------------------------------------------------------------------------------------

\begin{question}

    \questiontext{Consider the real capacitor modeled in Fig. \ref{fig:cir4}. }

    \begin{figure}[H]
        \centering
        \includegraphics[scale=1.5,angle=0]{Fig/cir4.pdf}
        \caption{Real capacitor model.} \label{fig:cir4}
    \end{figure}

    %--------------------------------------------
    \begin{subquestion}{Assume that the sinusoidal voltage $v_C(t)=A\cos(\omega t+\theta)$ is applied to the capacitor. Find the corresponding capacitor current $i_C(t)$ for ideal values of $R_{EPR}$, $R_{ESR}$, and $L_{ESL}$ and show that it has sinusoidal form.}
        \answer{}
    \end{subquestion}

    %--------------------------------------------
    \begin{subquestion}{Use PSpice AC sweep analysis to plot the amplitude and phase of the capacitor current versus $f=\frac{\omega}{2\pi}$ when the capacitor voltage is $v_C(t)=\cos(\omega t+\theta)$ and $R_{EPR}$, $R_{ESR}$, and $L_{ESL}$ have ideal values.}
        \answer{}
    \end{subquestion}

    %--------------------------------------------
    \begin{subquestion}{Use PSpice parametric analysis to plot the amplitude and phase of the capacitor current versus $f=\frac{\omega}{2\pi}$ for the capacitor voltage $v_C(t)=\cos(\omega t+\theta)$ and different suitably selected values of $R_{EPR}$, $R_{ESR}$, and $L_{ESL}$. Discuss how the parasitic parameters $R_{EPR}$, $R_{ESR}$, and $L_{ESL}$ affect the performance of the capacitor. }
        \answer{}
    \end{subquestion}

\end{question}

\assignmentSection{Bonus Experiments}

%----------------------------------------------------------------------------------------
%	QUESTION 8
%----------------------------------------------------------------------------------------
\begin{question}

    \questiontext{Write a MATLAB/Python code to plot the Lissajous curve corresponding to the voltage signals $v_x(t)=A_x\cos(\omega_x t+\theta_x)$ and $v_y(t)=A_y\cos(\omega_y t+\theta_y)$.
    }

    %--------------------------------------------
    \begin{subquestion}{Use the developed code to plot sample Lissajous curves for various values of $A_x$, $A_y$, $\omega_x$, $\omega_y$, $\theta_x$, and $\theta_y$. }
        \answer{}
    \end{subquestion}

    %--------------------------------------------
    \begin{subquestion}{Assume that $\frac{\omega_x}{\omega_y}=\frac{m}{n}$, where $\frac{m}{n}$ is a fractional number. Discuss how the ratio $\frac{\omega_x}{\omega_y}$ can be found from the corresponding Lissajous curve?}
        \answer{}
    \end{subquestion}

\end{question}


%----------------------------------------------------------------------------------------
%	QUESTION 10
%----------------------------------------------------------------------------------------

\begin{question}

    \questiontext{Return your work report by filling the \LaTeX template of the manual. Include useful and high-quality images to make the report more readable and understandable.}

\end{question}

%----------------------------------------------------------------------------------------

\end{document}
